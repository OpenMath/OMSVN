\chapter{Project Results and Achievements}

The main achievement of the project has been to support and develop the
OpenMath community, giving rise to many new activities and
collaborations.  This work will be continued under the aegis of the
OpenMath Society, an independent organisation incorporated in Finland
and run by an Executive Committee elected by its members.  Three other
specific achievements should be mentioned.

\section{OpenMath 2.0}

The OpenMath 2.0 standard is a major step forward for the OpenMath
Community, as it allows OpenMath applications to take advantage of
recent developments in the XML world as well as allowing OpenMath to
interact with semantic web technologies in a well-defined way.

The introduction of the \emph{cdbase} property for symbols allows for
each symbol to have a unique URI, formed by concatenating the cdbase,
cd name and symbol name properties.  This has already been used by the
MONET project (\url{http://monet.nag.co.uk}) to represent OpenMath
symbols inside an OWL ontology.

The introduction of the \emph{foreign} constructor delivers, amongst
other benefits, improved integration with MathML since it allows e.g.~an
OpenMath object to be attributed with its MathML presentation form in a
natural way.  It also delivers improved error handling.

The \emph{role} property for symbols can be used to improve validation
and error checking of OpenMath documents.  The move to an abstract
specification for content dictionaries allows for richer definitions of
symbols to be created in contexts where it makes sense (for use by
deduction systems for example).  

A full list of all the changes can be found in Appendix F of the
document, available from the website.

\section{Content Dictionaries}

During the lifetime of the project a substantial number of new CDs have
been developed, covering areas such as units and dimensions, abstract
algebra and polynomials.   New tools for validating CDs have been
produced and deployed on the website.

\section{MathML 2.0}

The Consortium has made substantial contributions to two W3C
Recommendations: MathML~2.0 (February 2001) and MathML~2.0, Second
Edition (September 2003).  NAG provided one of the editors and Stilo,
INRIA and the International University of Bremen each provided a
principal author.  In addition the University of Cologne provided an
in-depth review of various drafts.

MathML is now supported by Mozilla, Netscape and Internet Explorer (via
the MathPlayer plug-in) providing a cross-platform mechanism for
rendering mathematics on the web.  In combination with XSLT (also
supported by all three browsers) this provides a mechanism for rendering
documents including OpenMath.
