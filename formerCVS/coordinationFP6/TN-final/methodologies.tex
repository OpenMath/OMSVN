\chapter{Methodologies}

In many ways OpenMath was ahead of its time in adopting XML as one of
its encodings before the first XML Recommendation was released by W3C.
At that time no generic off-the-shelf XML tools existed and so it was
decided that OpenMath would use a restricted subset of XML to make
writing OpenMath parsers, validators etc.~easier.  Today XML is very
much a standard technology and a wide variety of XML components and
tools exist, so it made sense to drop the previous restrictions in
OpenMath~2.0.  

OpenMath also predates the semantic web concept, although as a semantic
markup technology it has much in common with it.  In OpenMath~2.0 we
decided to create a standard URI for every OpenMath symbol, to allow
OpenMath to be used as a vocabulary in RDF and OWL applications.  We
also introduced an abstract definition for Content Dictionaries so that
in future CDs could be encoded using, for example, RDF.  Although some
experiments were done to develop prototypes it was felt that to
standardise on a particular mechanism was premature.  OMDoc, a mechanism
for specifying complete mathematical documents, is also a candidate for
creating future CDs.

The OpenMath communities continuing collaboration with and involvement
in the W3C Math Working Group has ensured that both OpenMath and MathML
remain compatible.  Although most implementations of MathML so far have
concentrate on its presentation part (in browsers for example) its
support for content markup and in particular the {\tt csymbol} element,
make it an attractive ``alternative XML encoding'' for OpenMath.  
