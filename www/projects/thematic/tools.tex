\documentclass[report,keylogo]{openmath}
\usepackage{url,amssymb}
\begin{document}
\title{OpenMath -- Guidelines for Tool Developers}
\author{Stephen Buswell}
\address{Stilo}
\author{James Davenport}
\address{Univ.Bath}
\author{David Carlisle}
\address{NAG}
\author{Mike Dewar}
\address{NAG}

\date{November 2002}
\def\arccot{\mathop{\rm arccot}}
\def\arctanD{\underbrace{\mathop{\rm arctan}\nolimits{}}_{\mathop{\rm Derive}\nolimits}}
\let\rule\endgraf
\def\C{\mathbb{C}}
\def\R{\mathbb{R}}
\def\K{\mathcal{K}}
\let\subsubsubsection\paragraph
\let\subsubsubsubsection\subparagraph
\setcounter{secnumdepth}{6}
\setcounter{tocdepth}{3}
\maketitle
\tableofcontents

\section {Introduction}

\subsection {Purpose of this Document}
The intention of this document is to provide a starting point for
developers creating new OpenMath applications or adding OpenMath
functionality to existing applications. It gives the user a high-level
view of some common tasks, and some issues which need to be considered
when building an OpenMath-aware system. It also indicates where many
useful resources (both information and software) may be found.

\subsection {OpenMath Overview} 

OpenMath is a standard for the exchange of semantically rich
mathematical objects between communicating applications. The standard
and other related documents and resources can be found at
\url{http://monet.nag.co.uk/cocoon/openmath}.  A useful introductory
tour can be found at:
\url{http://www.risc.uni-linz.ac.at/~ocaprott/pisa2002/index.html}.

OpenMath structures mathematics into Content Dictionaries (CDs) which
contain symbols with precisely defined mathematical semantics. Two
applications can communicate if they both implement the OpenMath
semantics of a given CD, or an agreed subset of that CD. For ease of
communication and negotiation, CDs may be grouped into CD Groups.  The
creation and use of CDs is discussed in more detail below.

OpenMath defines an abstract model for constructing mathematical
objects from combinations of these symbols, together with two concrete
interchange formats, one XML-based, one binary.

The conversion of an OpenMath object between an interchange
representation and the internal representation in a software
application is performed by an interface program called a Phrasebook.
The translation is governed by the Content Dictionaries and the
specifics of the application.



\subsection {OpenMath and MathML}

OpenMath has a close relationship with MathML, the W3C recommendation
for mathematics on the web, \url{http://www.w3.org/Math}.  MathML has
support for both presentation and content models of mathematical
expressions.  The presentation side is frequently used to provide a
rendering mechanism for OpenMath, which has no native rendering
scheme. Conversely, OpenMath can be used as an encoding mechanism to
extend Content MathML beyond its basic scope.

Where OpenMath and MathML overlap, i.e., OpenMath CDs exist containing
symbols which also have MathML content elements assigned, the
semantics of the symbols in the two representations have been aligned.
An OpenMath CD Group, the 'MathML Compatibility Group' has been
defined, containing the OpenMath forms of exactly those symbols which
occur in Content MathML. 

\section{Common Tasks}

\subsection{Rendering}
OpenMath editing tools (see \ref{editing})  provide native
OpenMath rendering facilities, however for most applications
rendering of OpenMath applications will be via translation of the
semantically oriented OpenMath object to a different form more
directly associated with a rendering technology.

\LaTeX\ and (Presentation) MathML are two obvious possibilities for
specifying mathematical rendering, however other possibilities also
exist. For example most computer algebra systems have methods of
rendering mathematical expressions which are encoded in the language
of that system, thus a \emph{phrasebook} which maps OpenMath objects
to a system such as Mathematica or Maple may be used to obtain a
rendering of the object in addition to its use to encode the semantics
of the expression in the language of the computational system. It may
be possible to have a phrasebook that is specifically tied to
rendering that may be used for a wider class of OpenMath objects as
for a rendering-specific phrasebook there is less need to implement
the detailed mathematical functionality expressed in the Content
Dictionary in the Computer Algebra system being used.

\subsubsection{Rendering via \TeX}


For rendering via \TeX\ one needs to use a stylesheet in some language
to convert to \TeX\ mathematical markup, from which dvi, postscript,
pdf and other formats may be easily derived. The stylesheet should
ideally be easily extensible (see \ref{extensibility}) and allow
presentation rules for symbols introduced in new Content Dictionaries
to be easily added. It is likely that any such translation  will require
XML encoded OpenMath as input, the binary OpenMath encoding may be
supported by the use of an OpenMath Application that reads the binary
input and outputs the object in the XML encoding.

One such stylesheet implementations is available as part of the OMDoc
project: \url{http://www.mathweb.org/omdoc/}. 


An alternative route to \TeX\ rendering is to first translate to
Presentation MathML as described below, and then to use a stylesheet
that converts Presentation MathML to \TeX.

\url{http://monet.nag.co.uk/cocoon/openmath/software/mml-files/index.html}
DSSSL stylesheet for MathML Render MathML to TeX or RTF (For Microsoft
Word)

\url{http://www.raleigh.ru/MathML/mmltex/index.php?lang=en} Vasil
Yaroshevich's MathML to \LaTeX\ stylesheets.


\subsubsection{Rendering via MathML}

There are now several available stylesheets aimed at converting
OpenMath to MathML, for example:


\url{http://www.mathweb.org/omdoc/pres-architecture.html} OpenMath -
MathML stylesheets from the OMDoc Project.

\url{http://www.cs.nmsu.edu/~bpalmer/cmml2om} OpenMath - MathML
stylesheets from Brian Palmer.

Once the MathML is encoded in presentation MathML, it may be viewed in
a stand alone MathML viewer (such as the HELM project's viewer,
\url{http://helm.cs.unibo.it/mml-widget}) or more commonly via a MathML enabled web browser.  The
current state of MathML rendering in browsers is tabulated at the
W3C's MathML pages \url{http://www.w3.org/Math/XSL}, but in brief,
Presentation MathML rendering is supported in Amaya, Mozilla and (from
Version 7) Netscape browsers. For Internet Explorer free (no cost)
``Behavior'' extension is available from Design Science
\url{http://www.dessci.com/webmath/mathplayer} which will render both
Content and Presentation MathML. There are some differences in the way
MathML is supported in these browsers, principally the lack of support
for Content MathML in Mozilla/Netscape implementation and the fact
that using the Behaviors interface in Internet Explorer requires some
non standard Object elements and Processing Instructions at the head
of the document. However a stylesheet is available
\url{http://www.w3.org/Math/XSL} which detects the client being used
and modifies the document accordingly. This allows XHTML+MathML
documents to be served and displayed without change on Amaya, Mozilla,
Netscape and Internet Explorer. By default, Internet Explorer's
security settings do not allow XHTML pages on one server to access
stylesheets on a different server, so it is recommended that authors
take a copy of this stylesheet and reference a local copy of the
stylesheet from their pages.



\subsubsection{Rendering via SVG}

A third option for rendering OpenMath would be to convert to the W3C's
Scalable Vector Graphics Language (SVG). As for \TeX, one could write
stylesheets directly mapping to SVG but a common idiom is to first map
to Presentation MathML. Using MathML as an intermediate format in this
way allows the layout rules for the symbols in a Content Dictionary to
be specified just once, as Presentation MathML, and then support for
rendering objects using that Dictionary is automatically obtained for
any rendering technology for which you have MathML support.

One product implementing MathML to SVG conversion is available from 
SchemaSoft: \url{http://www.schemasoft.com/MathML} .

OpenMath to SVG is also implemented in Nice:
\url{http://mainline.essi.fr}.

\subsection{Phrasebooks}

While it is possible to write a phrasebook from scratch, there are a
number of tools and components to make the task easier.  A phrasebook
has to do one or both of the following:
\begin{enumerate}
\item parse an OpenMath object and translate it to an appropriate
  internal (native) representation;
\item translate a native object into OpenMath.
\end{enumerate}
The process of translation requires both an understanding of the
semantics of a set of content dictionaries, and the semantics of the
native data structures.

There are a number of tools to help build phrasebooks.  The INRIA
libraries (in C, C++ and Java) provide, amongst other things, an API
which reads an OpenMath object from a stream or file and returns a
series of tokens.  These tokens may then be interpreted by a user's
code and built up into a suitable native object.  The latest versions
of the libraries are available from the OpenMath software page at
\url{http://monet.nag.co.uk/openmath/software}.

There is also a Java library from RIACA which provides a higher-level,
more object-oriented API in terms of the abstract OpenMath objects
defined in the OpenMath Standard.  So whereas the INRIA libraries read
an OpenMath encoding and return a series of tokens, the RIACA library
will return an instance of an OpenMath object.  This is often easier to
use from a programming point of view, but for large objects it is less
efficient since both the OpenMath and native representations must be
created in memory.  The library, along with examples of phrasebooks
constructed for both GAP and Mathematica, is available from the RIACA
OpenMath page at \url{http://www.riaca.win.tue.nl}.

A related tool from RIACA is their CD Editor.  While primarily a tool
for creating and editing content dictionaries, it will generate a Java
Codec for a phrasebook which is compatible with the RIACA library, so
making it easier to add support for new CDs to a phrasebook.  At the
time of writing a beta version of the editor is available from
\url{http://www.riaca.win.tue.nl/download/om/cd/editor}.

Ad-hoc phrasebooks have also been developed based on the XSLT
stylesheets described elsewhere, in particular for applications which
support the reading and/or writing of Content MathML.  This is only
really a practical approach for phrasebooks designed to handle a small
number of content dictionaries.

\subsection {Editing and Manipulating OpenMath objects}\label{editing}

OpenMath is a standard for the interchange of mathematical objects
between communicating applications, so in many applications there is
no direct interface between the user and the OpenMath objects.
However in some cases it is useful to create objects separately.
A number of OpenMath-aware tools and components exist.

\subsubsection {OpenMath Editing Tools}
\url{http://mainline.essi.fr/jome/jome-en.html} The JOME OpenMath
Editor.  A Java component.  It supports graphical display and editing
of mathematical expressions.


\url{http://aiki.ccaps.cs.cmu.edu/DownloadIndex.html} OMDoc emacs
mode for writing OMDoc XML documents.
OpenMath DTD-aware, with element-completion and
context-sensitive menus.

\url{mailto:sb@stilo.com} STARS a MathML / OpenMath editor applet
which can be integrated into other systems.  STARS uses the Webeq
MathML rendering applet to render mathematics in web pages. It takes a
TeX-like linear syntax for mathematical input and translates this into
MathML and OpenMath.  Input can also be given in MathML and OpenMath.
It has been used as a front-end to various OpenMath systems, including
COQ and the NAG multiple-integrator demonstrator.

\url{http://www.matracas.org} QMath, a tool to produce OpenMath and
OMDoc with a text-based input syntax

\url{http://www.maths.tcd.ie/~richardt/openmath/} LaTeX to OpenMath
translator demo


Another possibility is to use a MathML-aware Computer Algebra application
such as Maple (\url{www.maplesoft.com})
export the MathML and then use one of the MathML-OpenMath translators,
This conversion is likely to be more successful if the Computer Algebra
system exports Content mathML rather than presentation. 

\subsection{Integrating Mathematical Applications}

OpenMath was originally designed as a data exchange format for
mathematical applications, however it does not prescribe any
particular transport mechanism which should be used to move OpenMath
object from one system to another.  There is a growing list of
environments which do this, an up-to-date list of which can be found
on the OpenMath software page at
\url{http://monet.nag.co.uk/openmath/software}.  We describe a number
of them briefly here.

\emph{JavaMath} (\url{http://javamath.sourceforge.net}) is an
easy-to-use suite of software which enables existing computational
engines to be accessed from Java programs.  It works by wrapping the
engine in a server layer so that it can, in principle, be accessed
from anywhere on the internet.  All data is passed using OpenMath.

RIACA have a number of components for deploying computational engines as
servers, and a \emph{shell} which allows a variety of such engines to be
accessed from a single environment.  They are all available from
\url{http://www.riaca.win.tue.nl/products/index.html}.

A number of more comprehensive software environments for allowing
mathematical systems to interact, using OpenMath as a data
representation language, are under development.  These include:
\begin{itemize}
\item \emph{OpenXM} (\url{http://www.math.sci.kobe-u.ac.jp/OpenXM})
\item \emph{IAMC} (\url{http://icm.mcs.kent.edu/research/iamc})
\item \emph{MathWeb} (\url{http://www.mathweb.org})
\item \emph{LogicBroker} (\url{http://www.mrg.dist.unige.it/~nembo/project.htm})
\end{itemize}

\section {Issues}

\subsection{Extensibility}\label{extensibility}

OpenMath is inherently extensible, there is a small core set of
features, Application, Binding, Integers, Floating Point numbers,
etc., but the majority of the semantics of any OpenMath expression is
specified by reference to one or more Content Dictionaries. It is the
intention that these dictionaries should be created as needed to
encode the semantics of different areas of mathematics.

This extensibility means that if possible generic OpenMath tools
should be able to handle new Content Dictionaries, either
automatically (for example an OpenMath Validator might not require
anything other than the list of symbol names which it can extract
directly from the CD file) or by explicit parameterisation, for
example a stylesheet converting OpenMath to MathML might require
additional stylesheet modules specifying the rendering of each CD, but
it should be possible to easily incorporate all the modules required
for any particular OpenMath Object.

Of course some OpenMath applications are not intended to be general
and will explicitly declare the Content Dictionaries that they can
handle.  This will be the usual case for applications that implement
specific mathematical functionality. Even in these cases the
application should accept objects using any CD, but as explained in
the OpenMath standard, may use the standard error CD to report errors
for objects that it can not process.

\subsection {The XML and Binary OpenMath Encodings}

The OpenMath standard provides two concrete representations for
OpenMath objects: one binary and one XML-based. Standard libraries and
phrasebooks exist supporting each representation. The decision of
which to use is application-specific.

In general one can say that the binary encoding is more efficient to
store and transmit.

The XML encoding is more verbose, although some improvement for
transmission can be obtained using schemes such as gzip as OpenMath
XML is text-based and highly-repetitive.  XML is more compatible with
industry standard tools such as XSLT (see also below). It is also
compatible with emerging XML-based interface and protocol standards
such as WSDL (Web Services Definition Language) and SOAP (Simple
Object Access Protocol) which may provide a basis for mathematical
services on the web.

\subsection {Using Generic XML tools with OpenMath}

The OpenMath XML encoding is a restricted subset of XML, so any valid
OpenMath object is a well-formed XML document, and can be validated
using the OpenMath DTD and any validating XML parser. A non-normative
XSD Schema for OpenMath objects can be found on the OpenMath site and
can be used with standard schema-aware systems.  It should be borne in
mind that the DTD is fairly permissive, and validation at the XML
level does not guarantee that the object is a valid OpenMath object
according to the other conditions imposed by the standard.

As has been noted above under Rendering, this XML nature can be
exploited for rendering OpenMath objects in a browser using XSLT
stylesheets. XSLT transformations can also be used to translate from
the OpenMath encoding to simple text-based input formats for some
mathematical applications.

Generic XML editors can also be used to create OpenMath objects,
although the internal structure of an OpenMath object does not make it
very amenable to meaningful display in document-oriented systems.





\subsection{The Semantics of OpenMath Applications}
OpenMath applications will vary in the precise mathematical semantics
that they possess, and also in the level at which they process
mathematics. Some applications may be computer algebra systems or
theorem provers, others will be databases or text handling
applications. For the former class of applications, the question of
which OpenMath constructs they understand is closely linked to their
internal semantics. For the second, there is an interesting question
of how they cope with the extensibility of OpenMath: new Content
Dictionaries can be written at any time, and the application should be
able to grow with this. One solution is to build a parallel dictionary
to the Content Dictionary structure, so that an application that
translated OpenMath into ZZZ would have a family of files with names
such as \url{arith1.zzz}, which would contain, in a format known to
the application, the rules for converting the symbols in the
\url{arith1} Content Dictionary into ZZZ. While this is not the only
way of being extensible, experience has shown that hard-coding Content
Dictionaries into converters leaves major maintenance problems.
\subsubsection{Understanding and Using OpenMath Content Dictionaries (CDs)}

\begin{quotation}
  Content Dictionaries are used to assign informal and formal semantics
  to all symbols arising in OpenMath objects. $\ldots$ The application
  receiving the object may then recognize whether or not, according to
  the semantics of the symbols defined in the Content Dictionaries,
  the object can be transformed to the corresponding internal
  representation used by the application.  \cite{OMstd}
\end{quotation}

A few additional remarks need to be made.
\begin{itemize}
\item The first is that it is not so much a question of an
  ``application'' understanding a Content Dictionary, rather it is a
  case of the application and an associated phrasebook understanding
  the Content Dictionary. For example, it would be totally legitimate
  for a Pascal generator to claim to understand the {\tt transc1}
  Content Dictionary, even though the only trigonometric functions
  defined in Pascal \cite{IEEEPascal} are $\sin$, $\cos$ and
  $\arctan$, and no hyperbolic functions are defined, since the
  translator could convert hyperbolic functions in terms of $\exp$,
  their inverses in terms of $\ln$, and missing trigonometric
  functions in terms of the existing ones, though special cases would
  have to be coped with, as in the OpenMath $\arccot z$, which could
  be translated into Pascal as $\arctan\frac1z$, or probably needs a
  more complicated construct such as
\begin{quote}
  {\bf if} $z=0$ {\bf then} $\frac\pi2$ {\bf else} $\arctan\frac1z$.
\end{quote}
\item An application can only be expected to understand a Content
  Dictionary within the application's own intrinsic limitations. Thus,
  although
\begin{verbatim}
<OMS cd="arith1" name="plus"/>
\end{verbatim}
  is defined to be associative, a fixed-precision numerical processor
  cannot guarantee to observe this.
\item A key idea is that an application/phrasebook combination must
  not mis-understand a symbol in a Content Dictionary. For example,
  for the reasons stated in section \ref{arccot}, a Maple phrasebook
  which translated the OpenMath $\arccot$ into the Maple one would
  definitely have misunderstood that symbol.
\end{itemize}
\subsubsection{Private or Public Content Dictionaries}
Content Dictionaries are the main means by which OpenMath gains its
unique combination of extensibility and interoperability. This means
that most application designers will be forced to face the question:
\begin{quote}
  which Content Dictionaries should I use? In particular, what should
  be the balance between public Content Dictionaries (from
  \url{www.openmath.org}) and private ones that I invent myself? If I
  use private ones, should I keep them private, or submit them to the
  OpenMath society via
  \url{http://monet.nag.co.uk/openmath/cdfiles/contrib}?
\end{quote}
There is also an interesting half-way house, of using the public CD,
but attaching a private annotation, using the {\tt OMATTR} construct
\cite{OMstd}.
\par
It is difficult to lay down an absolute decision process for this, as
the issues depend both on the symbols one wishes to use and on the
intent of the application.  Instead, we will state some general rules,
and illustrate these issues by a variety of examples from ``definitely
public'' to ``definitely private''.  \subsubsubsection{Definitely
  Public}
\begin{rule}
  If the semantics of the concept you want are the same as the
  semantics of an existing symbol in a public CD, then one should
  clearly use that symbol.
\end{rule}
So, for example, the function $\sin:\C \rightarrow\C$ (see section
\ref{matrixsin} for more complicated signatures) should definitely be
encoded by
\begin{verbatim}
<OMS cd="transc1" name="sin"/>
\end{verbatim}
and it would be bad for interoperability to use anything else.
\subsubsubsubsection{{\tt arith1} or {\tt arith2}}\label{commutativity}
One special case of the above rule needs considering. Both the {\tt arith1}
and {\tt arith2} CDs contain the symbol {\tt times}. The difference is that
the one in {\tt arith2} is definitely declared to be commutative. There is
one general principle, but it has to be tempered with ``fitness for
purpose'' considerations.
\begin{rule}
If it would be wrong to apply the operation to non-commutative objects,
then one should use 
\begin{verbatim}
<OMS cd="arith2" name="times"/>
\end{verbatim}
\end{rule}
For example, a linear algebra system that implements Gaussian elimination
should, in principle, insist on {\tt arith2}, since Gaussian elimination
does not work on non-commutative objects. However, if the system is not
polymorphic, and only operates on, say, IEEE floating-point numbers
\cite{IEEE}, then these are known to be commutative, so it would be sane
for it to accept {\tt arith1}'s version of {\tt times}. This is an example
of what we mean by ``fitness for purpose'' considerations.
\subsubsubsubsection{{\tt transc1} or {\tt transc3}}\label{transc}
\def\arctanIII{\underbrace{\mathop{\rm arctan}\nolimits{}}_{\mathop{\tt
transc3}\nolimits}}
Here, by contrast, the choice is much clearer. The functions defined in
{\tt transc3} are multi-valued analogues of the functions in the {\tt
transc1} Content Dictionary, and therefore have different signatures. Most
of them therefore return infinite sets, e.g. $\arctanIII(0)=\{n\pi\mid
n\in{\bf Z}\}$. It is therefore unlikely that a general-purpose computer
algebra system would use these in their standard interface.
\subsubsubsection{The semantics are not quite right}
This can occur in several ways.
\subsubsubsubsection{The signature is wrong}\label{matrixsin}
One example of this would be the use of $\sin$, defined by the usual power
series $\sin x = x -\frac16x^3+\frac1{24}x^5+\cdots$, but acting on, and
returning, square matrices. This is definitely not within the scope of the
signature from {\tt transc1.sts} \cite{STS} for
\begin{verbatim}
<OMS cd="transc1" name="sin"/>
\end{verbatim}
There are essentially three possibilities.
\begin{enumerate}
\item Ignore the issue, and use
\begin{verbatim}
<OMS cd="transc1" name="sin"/>
\end{verbatim}
\item Use the public symbols, but add a private attribute to keep track of
the difference, as in
\begin{verbatim}
<OMATTR>
  <OMATP>
    <OMS cd="mytypes" name="type"/>
    <OMS cd="mytypes" name="matrixvalued"/>
  </OMATP>
  <OMS cd="transc1" name="sin"/>
</OMATTR>
\end{verbatim}
By the standard rules on attributes, other systems that did not know about
the {\tt mytypes} CD could treat this as in the previous case.
\item Use a completely different CD, as in
\begin{verbatim}
<OMS cd="matrixtransc" name="sin"/>
\end{verbatim}
\end{enumerate}
Which to use depends on the context and the r\^ole of the application. We
give a few examples.
\begin{itemize}
\item An electronic book on matrices. If there is no computational support
for the chapter on elementary functions of matrices, then it would be sane
to use option 1, though if the OpenMath is going to be searchable, then
option 2 might be better, to distinguish this use of $\sin$ from the
$\C\rightarrow\C$ one.
\item A dedicated system for algebra and arithmetic with elementary
functions on matrices. Since no-one would be calling this system unless
they already knew that elementary functions on matrices was the name of the
game, option 3 would seem to be the best.
\end{itemize}
\subsubsubsubsection{The semantics are subtly different}\label{arccot}
We will illustrate this with two examples from the inverse elementary
functions: see \cite{CDJW}.
\begin{description}
\item[$\arccot$]The branch cuts for $\arccot$ have been a subject of some
controversy: see the table~\ref{arccot}.\begin{table}[t]

\vspace*{-\baselineskip}
\small

\caption{Different values of $\arccot(-1)$}
This table is taken from \cite{CDJW}, with the addition of the Maxima line.
\par\noindent
\begin{tabular}{llrl}
Source&Detail&$\arccot(-1)$&Comments\\
\cite{AS}&1st printing&$3\pi/4$&inconsistent\\
\cite{AS}&9th printing&$-\pi/4$\\
\cite{GR}&5th edition&?&inconsistent\\
\cite{CRC}&30th edition&$3\pi/4$&inconsistent\\
Maple&V release 5&$3\pi/4$\\
Axiom&2.1&$3\pi/4$\\
Mathematica&\cite{Mma}&$-\pi/4$\\
Maxima&5.5&$-\pi/4$\\
Reduce&3.4.1&$-\pi/4$&in floating point\\
Matlab&5.3.0&$-\pi/4$&in floating point\\
Matlab&5.3.0&$3\pi/4$&symbolic toolbox
\end{tabular}


\noindent
The note ``inconsistent'' means that, although the source quotes, or
clearly lets be inferred, a value for $\arccot(-1)$, there are enough
inconsistencies in the definition of $\arccot$ that one could infer a
different value. For \cite{GR}, $3\pi/4$ and $-\pi/4$ are equally
inferable.

\smallskip

\hrule

\mbox{}
\end{table} OpenMath uses the definition in
\cite[ninth printing]{AS}. This clearly poses problems for computer algebra
systems such as Maple and Axiom. The solution for the phrasebook for such
systems advocated in \cite{CDJW} is as follows.
\begin{description}
\item[OpenMath$\rightarrow$System]The OpenMath $\arccot z$ should be
translated into Maple as $\arctan\frac1z$, or possibly a more complicated
construct such as 
\begin{quote}
{\bf if} $z=0$ {\bf then} $\frac\pi2$ {\bf else} $\arctan\frac1z$. 
\end{quote}
\item[System$\rightarrow$OpenMath]This is distinctly more difficult, if the
system has generated some $\arccot$ expressions itself (it should never see
them on input, of course). The best translation is $\arccot z \mapsto
\frac\pi2-\arctan z$.
\end{description}
\item[$\arctan$]Derive's definition of $\arctan$ is subtly different from
that of \cite{AS}: differing only on the branch cuts. The definition of
\cite{AS} has the following relation with $\arcsin$:
\begin{equation}\label{arcsintan}
\arcsin z = \arctan\frac z{\sqrt{1-z^2}} +\pi\K(-\ln(1+z))-\pi\K(-\ln(1-z)),
\end{equation}
where $\K$ denotes the unwinding number \cite{CDJW}.
Derive has a different definition of $\arctan$ to eliminate the unwinding
numbers from equation (\ref{arcsintan}), so that, for Derive,
$\arcsin(z)=\arctanD\frac
z{\sqrt{1-z^2}}$. This definition can be related to that of OpenMath via
\begin{equation}\label{arctanD}
\arctanD(z)=\overline{\arctan\overline{z}}.
\end{equation}
This representation has the advantage that the rule
$\overline{\overline{z}}\rightarrow z$ means that the transformation is
self-inverse. Hence we could argue for the following transformations.
\begin{description}
\item[OpenMath$\rightarrow$Derive]Apply equation (\ref{arctanD}) to every
occurrence of OpenMath's $\arctan$.
\item[Derive$\rightarrow$OpenMath]Apply equation (\ref{arctanD}) to every
occurrence of Derive's $\arctan$.
\end{description}
However, if we are only using Derive as an integrator over $\R$, and if we
know that it will only return real $\arctan$s (rather than canceling
complex ones) then we need not do any transformations, since $\arctan$ and
$\arctanD$ agree as (partial) functions $\R\rightarrow\R$.
\end{description}
\subsubsubsubsection{The semantics are very different}
One example of this is the concept of approximate $\gcd$ of polynomials. The
ordinary symbol
\begin{verbatim}
<OMS cd="poly" name="gcd"/>
\end{verbatim}
is defined to be {\tt nassoc} \cite{STS}, which means that it takes an
arbitrary number of arguments, and is associative on them, so that
\begin{equation}\label{gcdassoc}
\gcd(\gcd(a,b),c)=\gcd(a,b,c)=\gcd(a,\gcd(b,c)).
\end{equation}
While there is (now) general consensus on the overall shape of a definition
of $\epsilon-\gcd$, {\it viz.\/} that an $\epsilon-\gcd$ of $p_1,\ldots,p_n$
is a polynomial $q$ of highest degree such that there exist polynomials
$\delta_1,\ldots,\delta_n$ of size less than $\epsilon$ such that
$$
q=\gcd(p_1+\delta_1,\ldots,p_n+\delta_n),
$$
there is less consensus on on the details, in particular the definition of
size, which can be, for $\delta=a_kx^k+\cdots a_0$, any of:
\begin{description}
\item[$||\cdot||_\infty$:]$\max(|a_0|,\ldots,|a_k|)$;
\item[$||\cdot||_2$:]$\sqrt{\sum_{i=0}^k a_i^2}$;
\item[$||\cdot||_1$:]$\sum_{i=0}^k|a_i|$.
\end{description}
While there are in principle the same three choices of symbol as in section
\ref{matrixsin}, the choice is much simpler. Option 1 is definitely wrong:
there is no way of specifying $\epsilon$, and the semantics are totally
wrong: the operation of $\epsilon-\gcd$ is not associative, and in some
formulations not commutative \cite{Rupprecht}. Option 2 could solve the
first problem, in a formulation such as 
\begin{verbatim}
<OMATTR>
  <OMATP>
    <OMS cd="mytolerance" name="L2Norm"/>
    <OMA>
      <OMS cd="arith1" name="power"/>
      <OMI> 10 </OMI>
      <OMI> -6 </OMI>
    </OMA>
  </OMATP>
  <OMS cd="poly" name="gcd"/>
</OMATTR>
\end{verbatim}
but it does not solve the more fundamental problem. Again, in the
``electronic book'' context described in subsection \ref{matrixsin},  one
could argue for option 2, but here there is the disadvantage that a
renderer might not know about the {\tt mytolerance} CD, and would therefore
drop the tolerance, so that the rendered result would appear
indistinguishable from an ordinary $\gcd$. However, the thesis of these
guidelines is embodied in the following rule, which strongly suggests
option 3.
\begin{rule}
If the mathematical and syntactic properties such as commutativity and
associativity defined in the formal mathematical properties and STS file no
longer hold for the public symbol in the new context, then a new symbol
must be used.
\end{rule}


\section{Conclusion}
This document surveys some of the issues relating to building an OpenMath
application, and has links to several example applications.
A more extensive list, which is updated as new software is announced
is available from the OpenMath web site at
\url{http://monet.nag.co.uk/openmath/software}.
This page should be consulted for any updates to the software mentioned
in this document.

If you do produce new OpenMath software it may be announced on the
public mailing list \texttt{om-announce@openmath.org}. Information on
subscribing to this mailing list is also available from the OpenMath web site 
\url{http://monet.nag.co.uk/openmath/lists}.

\let\oldsec\section
\def\section*{\oldsec}
\begin{thebibliography}{9}
\bibitem{AS}
Abramowitz,M. \& Stegun,I.,
Handbook of Mathematical Functions with Formulas, Graphs, and Mathematical
Tables.
US Government Printing Office, 1964.
10th Printing December 1972.
\bibitem{CDJW}
Corless,R.M., Davenport,J.H., Jeffrey,D.J. \& Watt,S.M.,
``According to Abramowitz and Stegun''.
{\it SIGSAM Bulletin\/} {\bf34} (2000) 2, pp. 58--65.
\bibitem{STS}
Davenport,J.H.,
A Small OpenMath Type System.
{\it ACM SIGSAM Bulletin\/} {\bf34} (2000) 2 pp. 16--21.
\bibitem{IEEEPascal}
IEEE Standard Pascal Computer Programming Language.
IEEE Inc., 1983.
\bibitem{IEEE}
IEEE Standard 754 for Binary Floating-Point Arithmetic.
IEEE Inc., 1985.
Reprinted in {\it ACM SIGPLAN Notices\/} {\bf22} (1987) pp. 9--25.
See also
\url{http://www.cs.berkeley.edu/~wkahan/ieee754status/754story.html}.
\bibitem{GR}
Gradshteyn,I.S. \& Ryzhik,I.M. (ed A. Jeffrey),
Table of Integrals, Series and Products.
5th ed., Academic Press, 1994.
\bibitem{OMstd}Caprotti,O., Carlisle,D.P. \& Cohen,A.M., The OpenMath
standard. The OpenMath Consortium, 2000.
\url{monet.nag.co.uk/openmath/standard}.
\bibitem{Rupprecht}
Rupprecht,D.,
{\it \'El\'ements de G\'eom\'etrie alg\'ebrique approch\'ee~:
\'Etude du PGCD et de la factorisation.}
PhD thesis, Universit\'e de Nice--Sophia Antipolis, Jan. 2000.
\bibitem{Mma}
Wolfram,S.,
{\it The Mathematica Book\/}.
Wolfram Media/C.U.P., 1999.
\bibitem{CRC}
Zwillinger,D. (ed.),
{\it CRC Standard Mathematical Tables and Formulae\/}.
30th. ed., CRC Press, Boca Raton, 1996.

\bibitem{issac}
S. Dalmas, M. Ga{\"e}tano and S. Watt,
{\it An OpenMath 1.0 Implementation}.
Proceedings of ISSAC 97, ACM Press 1997.
\end{thebibliography}

\end{document}





