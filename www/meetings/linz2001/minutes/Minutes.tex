\documentclass[11pt, a4paper]{article}
\usepackage{url}

% a bit wider
\advance\evensidemargin-.5in
\advance\oddsidemargin-.5in
\advance\textwidth1in

\begin{document}
\title{OpenMath Meeting 26--28 September 2001\\
RISC, Hagenberg, Austria\\}
\author{Minutes by James Davenport \& David Carlisle}
\maketitle
\section{Joint OpenMath/MKM day 26.9.2001}
\subsection{The NIST Digital Library of Special Functions}
This talk was given by Daniel Lozier of NIST. He emphasised that this
project was {\it not\/} about taking the 36-year old\footnote{Or even more:
much of the information was collated by Abramowitz before his untimely
death in 1958.} information in the book and placing it on the web. The
traditional Abramowitz and Stegun is but one of many sources going into the
new information source. The publication date is expected to be in 2003. He
outlined the subject matter as being:
\begin{itemize}
\item special functions of applied mathematics;
\item validated technical data;
\item to meet proven needs in physics and other sciences (i.e.~not things
only useful within mathematics);
\item chapters on individual functions;
\item methodology chapters (numerical methods, analytical methods,
asymptotic methods and computer algebra);
\item indexes and search engine.
\end{itemize}
The format will be a book {\it and\/} a search engine.
The current web site is \url{http://dlmf.nist.gov}.
\par
A typical chapter's contents would be
\begin{itemize}
\item mathematical notation and properties --- the original Abramowitz and
Stegun had a powerful normative influence on notation;
\item metadata;
\item graphs and visualisations;
\item sample applications;
\item computational methods;
\item pointers to software (commercial and non-commercial);
\item references.
\end{itemize}
The style will be terse ``handbook'' style, aimed at experienced scientists,
rather than a traditional mathematical or pedagogical style. The book will
be about 1000 pages, roughly the same size as the original, but containing
twice as much mathematics, since the tables of the original will not be
necessary.
\par
There is a special \LaTeX{} class for the DLMF, with macros, equation
breaking, double (largely for the printed form\footnote{There is also a
printed form with one column of text and one column of metadata, largely
for use by authors and validators.}) and single (mostly
on-screen) column style, and optional editing forms. This is accompanied by
a special version of \verb+latex2html+.
\par
The metadata has two purposes.
\begin{enumerate}
\item Author's notes to help the reader: proof hints, additional
references, notational reminders and acknowledgements.
\item Indexing metadata to construct both paper and Web indices, and to
drive the search engine. One index will cross-reference the original
Abramowitz and Stegun equation numbers to the new format.
\end{enumerate}
The metadata is generally tied to the subsection, rather than the
individual equation.
\par
The speaker gave a guided tour of the Airy functions chapter\footnote{This
was the sample chapter produced in 1998.} in various
formats.  One new section is that on error bounds, for both real and
complex values, for the various asymptotic expansions presented. Recent
work on exponentially-improved asymptotic expansions is also included.
Under graphs and visualisations, there are line plots, contour plots,
diagrams, and (colour) surfaces. It is unlikely that there will be many
animations. However, the colour surfaces, which are largely for complex
functions, are full VRML surfaces, and can therefore be rotated,
intersected with various coordinate planes, etc. A very graphic
presentation of the branch cut of a Hankel function was given.
\par
The goal of the search engine is to be able to search in equations, which
is not supported by current technology. There is a partial solution based
on metadata and a very extensive thesaurus, together with a ``pidgin
math'' parser.  It is possible to search for \verb+Gamma(1/3)+,
\verb=Ai^2+Bi^2=, \verb+b_$+ (a subscripted $b$, with the subscript
unspecified) and so on. The engine is parameterised by a ``search depth'',
which will, for example, find $Ai(z)^2+Bi(z)^2$ as an answer to the second
query.
\par
Dr. Lozier also spoke to the presentation that Bruce Miller would have
given. He pointed out that the distinction between parameters and arguments
is somewhat artificial, but only somewhat. The markup is
\verb+\BesselJ{\nu}+ or \verb+\BesselJ{\nu}(z)+. There is much more known
about hypergeometrics than in the days of Abramowitz and Stegun, where
there are essentially three kinds of arguments:
\begin{verbatim} 
\hyperpFq{p}{q}@{a_1,\ldots}{b_1,\ldots}{z}
\end{verbatim} 
might be used.
\subsection{Certified and Portable Mathematical Documents}
Martijn Oostdijk presented joint work with Olga Caprotti and Hermann
Geuvers arising out of the ``Algebra Interactive!'' project.  He pointed
out that, within a mathematical document, there are many links between
definitions, theorems, proofs etc., as well as within proofs, thus making
these ideal for hyperlinks. In COQ's CIC, both objects and proofs inhabit
the same universe, thus meeting de Bruijn's criterion and allowing a small
typechecker. He noted that it was necessary to produce a ``COQtree'' in
Java, and that the HEML  project had presented a neat way of doing this via
XML. Given a type-annotated COQtree, we can produce an OMdoc tree, which
can be converted to an XML document, which an XSLT stylesheet can convert
to the HTML which is actually displayed.
\par
They had improved Coscoy's rendering of COQ into ``natural language'', by
sometimes created sub-documents to reduce the amount of nesting that would
otherwise be involved. He then gave a demonstration of this.
\subsection{Mathematical Knowledge Representation}
James Davenport presented this thesis, that knowledge representation was a
vital pre-requisite for mathematical knowledge management.
\par
Michael Kohlhase pointed out that the algebraic specification community,
unfortunately not well represented at MKM, had answers to many of the
questions. JHD agreed partially, but said that they did not have all the
answers.
\subsection{Meta stylesheets for the conversion of mathematical documents
into multiple forms}
Bill Naylor presented joint work with Stephen Watt on this subject.
The problem being addressed is that of converting (extended) MathML to
other forms, caused by the many$\leftrightarrow$many correspondence between
presentation and semantics. One example is the many forms of the binomial
coefficient, where $n \choose m$ can also be represented as ${}_mC^n$, or,
worse, because totally ambiguously, $C_m^n$ or $C_n^m$. They therefore
claimed that one could store, either in the content dictionaries or in a
parallel system analogous to the \verb+.sts+ files, notational information.
\par
They therefore proposed a \verb+<Notation>+ element. This could have
(several) \verb+<version>+ elements, one for each possible presentation of
a semantic concept, which could, say, be \verb+<math>+ (presentation
MathML), \verb+<tex>+ (for \LaTeX) or \verb+<image>+ (an URI to a \verb+.gif+
file). There would also be matching semantic templates, as in
\begin{verbatim}
<semantic_template> <OMOBJ> <OMA>
  <OMS cd="combinat4" name="choose"/>
  <OMV name="n" id="argChoose1"/>
  <OMV name="m" id="argChoose2"/>
</OMA> </OMOBJ> </semantic_template>
\end{verbatim}
where the \verb+id+ tags are used as cross-references in the \verb+<version>+
elements to the various arguments. There could be more than one
\verb+semantic_template+ elements, since, say, integration would need
different forms depending on whether it was $\int_a^b f(x)\,{\rm d}x$ or
$\int_{x\in S} f(x)\,{\rm d}x$. It is also necessary to have template
functions, e.g. to compute $n$ in the notation $\frac{\partial^n}{\partial
x^2\partial y^3\ldots}$.
\par
In this system, content dictionaries, or there associated files, could be
processed by a ``meta'' style sheet that would produce various combined
stylesheets for converting to various formats, such as straight
presentation MathML, or presentation MathML with \verb+csymbol+ OpenMath
references, or $\ldots$.
\par
They suggested that, say, \verb+<xmml:choose style="2"/>+ could be used to
choose one of the \verb+<version>+ elements from the \verb+<Notation>+
elements mentioned above. If this wasn't given, then information could come
from the defaults in the ``meta'' style sheet, or in the content
dictionaries themselves. 
\par
He concluded by saying that the OpenMath Content Dictionary concept could
provide a useful carrier for notational information.
DPC pointed out that it might be necessary to have the full XSL
functionality (as in \verb+xpath+) to select sub-sub-arguments etc. His example
was that of $\int\lambda x.\sin x$, which should be rendered as $\int\sin x
\,{\rm d}x$. Sacerdoti Coen was worried about the performance implications
of a large XSL stylesheet.
\par
AMC asked whether this wasn't simply a phrase-book.
MK thought that this approach was more flexible than a simple phrase-book
approach.
\subsection{Mathematical Software: the Next Generation}
Mike Dewar presented a joint paper with David Carlisle on this topic. He
began by summarising the history of NAG and mathematical software. The
``temporary'' Fortran library written to tide it over until the rise of
Algol was still going thirty years and twenty versions later.
\par
He noted that the scientific software market was very conservative. It was
also very heterogeneous: research users; production users and
education/training users. Take-up of web-based services would be very
different in different segments.
\par
He summarised the general wish for ``plug and play'' mathematics via the web.
He then listed various problems.
\begin{itemize}
\item Semantics of objects and software.
\item Handling errors and exceptions in a distributed environment.
\item Tracing and debugging in a distributed environment.
\item Embedding existing software in an environment it was not designed
for.
\item Scalability (partly of problem size, but also of the software life
cycle).
\item Reliability and reproducibility.
\item Commercial and licensing issues (not today's issue, but a very real
one).
\end{itemize}
The outline strategy was to develop  a framework for embedding software
components automatically, based on abstract specification: software
``glue''; documentation; testing and verification material. There was also
a need for resource discovery, based on standards such as RDF and WSDL.

He said that JHD had already spoken about OpenMath Content Dictionaries, but
pointed out that OpenMath provided lightweight mechanisms for semantics.
\par
He pointed out that the NAG Fortran library had 1200 user-callable
routines, documented in 100Mb of PDF with 96,000 mathematical expressions.
This was, unfortunately, unstructured, presentation-oriented and
Fortran-specific, with limited verification and validation against the
software. Nevertheless, he claimed that this was a significant repository
of mathematical knowledge.
\par
Many component frameworks are based on the Interface Description language
(IDL). This is not powerful for mathematics. NAG therefore has an
extension, which specifies for parameters: intent, purpose (argument,
control, workspace, array dimensions), concrete versus
abstract\footnote{such as ``upper triangular matrix''.} type, defaults,
constraints and specifications. This was semi-automatically derived from
the existing documentation, and is represented in XML.
\par
He then spoke about NAG's ideas on mathematical services. We need to be
able to describe these, and support automatic service discovery. This needs
\begin{itemize}
\item problem analysis;
\item qualitative decisions;
\item explanation and justification mechanisms;
\item mathematical and non-mathematical criteria;
\item refinement of decisions in the light of experience and/or failure.
\end{itemize}
His example was the computation of $\int_0^1\sin x\,{\rm d}x$, which could
be done by \verb+D01AJF+, or via the removable singularity routine
\verb+D01AHF+, or the sine-multiple routine \verb+D01ANF+, or conceivably
the sine-integral routine \verb+S13ADF+. How is this to be elicited?
\subsection{Likely Opportunities in Framework 6}
Herr Hans-Georg Stork\footnote{European Commission offices, Luxembourg.
\tt Hans-Georg.Stork@cec.eu.int} spoke to this, using a mathematical analogy.
\begin{description}
\item[Definition]The biggest pillar of FP6 will be ``integrating European
research''. The other two are ``Strengthening the ERA'' and ``Supporting
the ERA'' --- ERA = ``European research Area''. Within this biggest pillar,
there are various ``priority thematic areas'', and ``anticipating Science
and Technology needs''. One priority thematic area is ``Information Society
Technologies''.
\item[]Two of the major instruments of FP6 will be ``Networks of
Excellence'' (anticipated to have more funding than current networks, say
8--15 MEuro) and ``Integrated Projects'', which certainly have to include
technical development, and might be 20--100 MEuro. There will also be room
for the equivalent of the current IST projects.
\item[Theorem]The probability that MKM will be included in FP6 looks very
high.
\item[Proof]This is divided into two cases.
\begin{description}
\item[(MK)M]Management of Mathematical Knowledge. Past projects have been
OpenMath and EULER. The OpenMath Thematic Network is alive.
\item[M(KM)]Mathematics of Knowledge Management. Past projects 
include IBROW, which used ontologies and ontology-based reasoning. Projects
such as Ontologging, OntoKnowledge XML-Knowledge management etc. are alive.
In the intersection, there are projects such as Calculemus and Types.
\end{description}
He noticed the amount of information available on Cordis under the keywords
``mathematics'' and ``knowledge''.
\item[]There is the current call under ``Semantic Web Technologies''. One
line is ``creating a useful formal framework'', which is very relevant to
MCD's talk. 
\item[]The FP6 IST draft includes a box called ``knowledge technologies'',
which he expects to be quite substantial.
\item[Corollary]He expects that both M(KM) and (MK)M will be funded. He
recommended \url{http://www.cordis.lu/rtd2002} for the state of the
discussion on FP6, and \url{http://www.cordis.lu/ist/ka3} for the current
FP call.
\end{description}
Andrzej Trybulec asked about the status of Poland. It was hoped that
Poland would be fully included in FP6.
Michiel Hazewinkel said that there would be an MKM project-planning meeting
in Amsterdam in November, and invited HGS to it.
\subsection{Modelling for Understanding of Scientific Knowledge}
Saverio Solerno gave this talk, which is situated in the framework of
e-learning and intelligent tutoring systems. The aim was to set forward a
representation method for a domain (e.g. calculus) in which points such as
inductive or interdisciplinary ones can be considered as well as the
hierarchical deductive points. At this stage, they are thinking of a fixed
domain of knowledge.
\par
It is important that the system understands the user's strengths and
weaknesses, and can have a model of his misconceptions. The mathematical
model used is that of a multigraph, whose nodes are the atomic concepts of
the domain.
\subsection{Contribution of Ontology Engineering to MKM}
Francky Trichet gave this talk. The GINA project deals with sketches and
sentences expressed in natural language (constraints etc.). The system
includes geometric knowledge to interact with the user and to handle
queries. This knowledge of projective geometry must be formally represented
in the system. The knowledge acquisition process can be viewed as a corpus,
being transformed by conceptualization to a conceptual model, which is
transformed by ontologization to an ontology, which is then given a formal
representation. He saw a semi-formal ontology as having two components: a
formal part with a clear and consensual semantics, and an informal part
which did not have a consensual semantics. In this case, the corpus was
Hilbert's ``Grundlagen der Geometrie'', where the knowledge is already
conceptualized: concepts (point, straight line, plane etc.), relationships
(membership etc.) and axioms.
\par
This is to be represented in the Conceptual Graphs model. This has two
levels of representation: terminological (concept types and relation types)
and assertional (representation of facts with conceptual graphs etc.).
There is a hierarchy of concepts: straight line $\in$ flat curve $\in$
affine curve $\in$ set of points etc. and a hierarchy of relationships.
There is a projection from a graph $G$ into a graph $H$ if $G$ is more
general than $H$. A rule $R$ is applicable to a graph $G$ if there is a
projection from the hypothesis of $R$ to the graph $G$, and then the
conclusion of $R$ can be added to $G$. Constraints are a pair of graphs,
which can be negative (if $A$ is present, then $B$ must be absent) or
positive (if $A$ is present, then $B$ must be present). In this case, there
are:
\begin{itemize}
\item 5 negative constraints, such as the incompatibility of $\in$ and
$\notin$;
\item 17 rules representing axioms;
\item 10 rules representing implicit knowledge;
\item 2 definitions of relation types;
\item 1 definition of concept types.
\end{itemize}
Such a conceptual model can be used in knowledge management and in
automatic theorem proving.
\par
It was asked whether a description logic language could not be used
instead. Buchberger asked if set theory and first-order logic were not
sufficient for this.
\subsection{Panel discussion}
The panel was chaired by Michiel Hazewinkel, and included all the
participants. Buchberger reminded the meeting that the idea for this MKM
workshop came from a workshop where he was invited to Amsterdam by MH.
Various questions had been circulated, and some were addressed by the
panel.
\begin{enumerate}
\item Monolithic or modular? Farmer said that the system he envisaged in
his talk was not necessarily monolithic, but that the ultimate system would
have to address the whole of mathematics. MK had not envisaged Farmer's
system as monolithic, rather as the specification of a protocol. He would
like to see a distributed system. Baumgartner said that his system need not
be web-based, and was not necessarily intended to model mathematics from
its foundations. Ion said that one could not have a monolithic system for
mathematics: there is just too much of mathematics. Buchberger said that it
was much too early to conceive of a monolithic system: many different
approaches needed to be explored. Even the question of which logic to use
was not obvious. Here he saw the distinction between the formal system and
the foundational system. It was not always necessary to go back to first
principles: in his project it would be unrealistic to reduce the whole of
Hilbert space theory to set theory. It was stated that there were various
needs for knowledge management: experimentation, formal verification etc.,
and this found wide-spread approval. Constantini (Rome) said that it was
necessary to have some framework for communication. Trybulec said that,
for every Mizar article, there was an (information-theoretic) ancestor.
Though there were over 7,000 articles, the maximum height of this tree was
12. Farmer summarised by saying that people should be able to use multiple
logics, but there needs to be some way of communicating between them. MH
said that one could waste an enormous amount of time making sure that all
the systems could communicate with each other, and we had to do something
about ``approximate communication''.
\item ``Is an MKM feasible? The MKM conversion factor is likely to be 4 or 5
(analogous to the de Bruijn factor).'' Ion said that the answer was to
start, in order to improve the technology. MK said that Psyc project, in
A.I., was a pretty miserable failure, since (a) the methods were not ready,
and (b) the attempt was monolithic. He was more optimistic about the
current activity. Borwein asked what the boundaries of mathematics were
defined to be. MH said that this was a difficult question. Borwein said
that we needed to formulate some realistic sub-goals. MH said that we ought
to manage most that part of mathematics which was most used. Special
functions probably fell into this category. MK said that we should not aim
at being complete. AC said that we should try, at least, to standardise the
notations. HGS asked precisely what Knowledge Management was, and why was
Mathematical Knowledge Management special. MH
said that several of the talks had addressed this, partly giving the reason
that mathematics was well-structured. Ion said that the world assumes that
mathematics is well-defined, so this was a good test case. There was a call
for stating precisely what the issues of MKM were, at the mathematical,
logical and communication levels.
\item[6.]``Formal proofs are fragile --- Caldwell.'' A slight change to a
theorem prover can mean that previous proofs no longer work. JHD suggested
that it was important to store some of the intermediate stages, e.g. the
output of the tacticals. Buchberger said that some informal, human, proofs
were also fragile. MK thought that fragility was largely a technology
problem, and a failure of knowledge management. Borwein commented that the
fragility of human proofs depended on the size of the user community. AMC
commented that there was, in fact, no proof of the classification of finite
simple groups, but everyone believes it. Hardin said we had to distinguish
between truth and confidence.
\item[7.]``Most mathematical knowledge is in the heads of mathematicians,
and is not written down --- Farmer''. Hardin asked if it was possible to
make this knowledge explicit. In France, there was a problem here in the
area of nuclear systems. JHD drew attention to Traverso's submission to
this conference, which drew attention to the amount of undocumented
knowledge in computational algebraic geometry. MH asked if this was unique
to mathematics. HGS said that one cannot learn mathematics from books
alone, which is a converse of this challenge, and solving this problem
would be solving a hard A.I. problem. Borwein said that recoiling from the
excesses of Bourbaki had made certain things more explicit than before.
\item[8.]``Who profits from MKM?'' Farmer said that most users of
mathematics need a very small amount of mathematics. MK thought that much
of what we were doing here carried over to Physics, or much of hard science
or engineering. This was where to find rich and receptive customers. 
\end{enumerate}
MH concluded by reminding all that there would be a meting in Amsterdam in
the middle of November. What tasks should be given to the group there? MK
thought that a convincing answer to point 8 was the key. AMC said that the
setup of OpenMath was flexible enough to be used as a basis for
communication. JHD said that he thought OpenMath was indeed powerful and
flexible enough for the communication of {\it mathematical objects\/}, and
the challenge was to convert that to the communication of {\it
mathematics\/}. HGS wanted an (optimistic) proof of the existence of an
infrastructure within the mathematical community to perform MKM.  HGS said
that answering question 8 should also link to other areas, such as the
Ontoweb Thematic Network.
\section{OpenMath Workshop 27.9.2001}
Approximately 20 people were present.
\subsection{MathML tools}
David Carlisle spoke to this topic. He said that MathML had now been around
for a while, so that it was appropriate to ask what tools had been
developed for it. The state of browsers was summarised as follows.
\begin{itemize}
\item Mozilla+MathML --- with a presentation MathML compile-time option
(provided that the font support is there: apparently a problem on the
Macintosh).
\item Mozilla+XSL+MathML --- will let one add content MathML. Still needs
the compile-time option. The XSL implementation is currently somewhat
suspect in Mozilla.
\item[]He noted that the MathML, which was not suspect, was not in the
Netscape version of Mozilla, whereas the suspect XSL support was.
\item Amaya supports presentation MathML.
\item Internet Explorer: Microsoft has said that they will not support
native MathML. Currently the mechanisms to download a plug-in/applet into IE
5.x are somewhat clunky, which is a serious problem for the take-up of
MathML. IE 6's support of XSL is currently broken, which is a problem.
\item[]{\it If\/} XSL, DOM support and Javascript work, it is possible to
simulate MathML support in IE: this should mean that documents do not need
to state which plug-in they require, and the style sheet can work out what
is available, and use DPC's simulation otherwise.
\end{itemize}
He discussed the state of Techexplorer and Webeq. Design Science (who make
Word's equation editor and MathType) were to have demonstrated at Linz: DPC
demonstrated this, and showed that one could cut from Mathtype and paste as
MathML.
\par
There is a problem with type-setting quality. There are three \TeX{} packages
that read MathML. It is also possible to use XSL (say) to convert MathML
into \TeX, though these are not quite as good as they could be, typically
due to the amount of hand-tuning in \TeX{} input.
Wolfram Research would claim that Mathematica is the way of typesetting
MathML, and this is viable.
\par
tex4ht and omega are DVI-based tools for converting \TeX{} into MathML:
since they are DVI-based, specialised mark-up (e.g. that used in DLMF) will
have been lost.
\par
In questions, PL asked for a MathML resources list, to contain the current
state of this information. The new W3C staff member for MathML (Max
Froumentin) is hoping to work on this.
MCD commenting that Maple 7 had support for downloading relevant applets.
PL asked for support in converting presentation (\LaTeX{} or presentation
MathML) into content MathML. Several people commented that you needed to
know the context. WAN mentioned the work at Western Ontario, which should
be appearing on their Web site.
\subsection{OMDoc in use}
Baumgartner spoke to this work, joint with Antje Blohm and Margret
Gross-Hardt. He was speaking from the customer's point of view. The In2Math
project is about electronic teaching material, where the student has access
to domain tools, e.g. a computer algebra system. One application at Koblenz
is in teaching logic. He emphasised that students might wish to use several
logic tools, so portability of formulae was important. There was also a
requirement for different styles of input, e.g. clausal logic, predicate
logic, as well as the STRIPS planning language.
\par
He complained about the readability of OpenMath, and the difficulty of tool
writing. OC pointed out that arity checking, etc., was possible, but not
yet implemented in the Java library. This became a debate on the r\^ole of
phrasebooks.
\par
After talking to Kohlhase, he thought that they could use the OMDoc
``presentation'' element to control associativity etc.
Hu summarised by saying that there were three options.
\begin{description}
\item[XML]intuitive, but home-made.
\item[OpenMath]standard, but little structure and tool writing was difficult.
\item[OMDoc]A practical meta-stylesheet approach, but still little
structure.
\end{description}
SB suggested a fourth possibility.
\begin{description}
\item[binary]Dissociate semantics from presentation, by writing a (XSL)
style-sheet to convert OpenMath to presentation MathML.
\end{description}
DC felt that it would be easy to map from any of these formats to any
other, so that the problems were fundamentally the same. PL pointed out
that there was also a problem of user input in this application.
\subsection{Electronic Books and OpenMath}
Reinaldo Barriero (Eindhoven) spoke to this title.
He said that MathBook, an XML application, should be capable to produce
Web applications (HTML pages, JSP pages, XML documents) and also \LaTeX, and
hence DVI and PDF. They had been using OMDoc, but found this too rich. He
defined a JSP page as a text-based document specifying how to process a
{\it request\/} to generate a {\it response\/}. This is claimed to be
``write once, run anywhere'', to separate the r\^oles of developers and
authors, to encapsulate functionality (JavaBeans and tag libraries). He
illustrated this with a sample mathematical page.
\par
We continued to describe the IDA tag library. This allows actions such as
interaction with back ends (currently GAP and Mathematica), transformations
(OpenMath $\rightarrow$ MathML), casting OpenMath objects (sets into lists,
lists of lists (as returned from Mathematica) into matrices), parsing into
OpenMath etc., flow control, and working with scopes (in the sense of
scopes across pages and within sections of the book) and variables. He
illustrated this with a use of a programming CD to send a Fibonacci program
to Mathematica. He claimed that this was much shorter than the JavaBeans
approach from the author's point of view.
\par
He summarised future work as being:
\begin{itemize}
\item more phrasebook work;
\item new CDs;
\item new tags;
\item efficiently producing good presentation MathML from OpenMath ---
essentially a MathML phrasebook\footnote{which might also mean a way of
giving presentation hints in OpenMath.};
\item more work on MathBook and its relationship to OMDoc.
\end{itemize}
PL asked whether JSP could be used to generate \LaTeX: the answer was
affirmative. OC asked where this information was available: it is currently
in private CDs. DC commented that JSP itself was not XML, and asked what
problems this caused. SB remarked that, unlike what had been shown, the
Java code could be external to the page, and stored in the bean, and there
is an XML encoding for enough JSP to support this. AMC commented that there
was an ``algorithm'' CD, and this meant that the same page could work with
GAP as well as Mathematica.
\subsection{Special Functions in OpenMath}
JHD spoke to this issue. He said that one unsolved problem was functions
defined by analytic continuations. Farmer said that there were three kinds
of definition.
\begin{enumerate}
\item An object defined by a formula, as for $\arctan$ in the
\verb+transc1+ CD.
\item An object defined uniquely by a set of properties, which might
incorporate JHD's \verb+odesolution+ and similar concepts.
\item An object specified by a set of conditions, but not necessarily
uniquely.
\end{enumerate}
The second class might subsume JHD's various worries about specification of
special functions. JHD said that he would like to work a few examples.
\subsection{Changes to the OpenMath standard}
DPC spoke to this issue. he said that there were various levels of changes.
\begin{itemize}
\item Textual corrections, e.g. \verb=http:www= $\rightarrow$
\verb=http://www=. There are other problems of these natures: would fixing
this result in a change of version number? The formal standard probably
ought not to contain the change log and marginal notes it currently
contains. So what is the formal reference version: \LaTeX{}, PDF or what?
He is now of the opinion that there should be a master XML source as the
normative version. Again, what implications for the versioning? AMC
commented that there had been discussions about version numbering. MCD
thought that these were about CD versioning, rather
than versions of the standards. JHD thought that the change of normative
language was a significant change, e.g. to 1.1, and there seemed to be some
consensus here. We would have to continue to distribute the PDF version.
\item MathML-related changes. Some might be errata (e.g. JHD's comments on
differentiation) and others might be implied by changes in MathML. In
theory, MathML might require changes in the standard, though this should be
avoided as far as possible.
\item AS's concerns in Berlin about binding symbols. The standard specifies
that binding multiple variables is equivalent to multiple bindings, which
can mean confusing variable capture.
\begin{itemize}
\item Leave it as is.
\item Specify that $B$ is a symbol.
\item Specify that $B$ has no free variables.
\item Specify that the free variables of $B$ exclude $v_i$.
\item Specify that this equivalence is only true after $\alpha$-conversion.
\end{itemize}
JHD commented that there were implications for his \verb+ODEsolution+
syntax. Some people thought that the comment on equivalence should be
deleted.
\item The formal definition of an \verb+OMF+ refers to the IEEE standard,
which is not very available. The view was that there was not much we could
do about this, as it was the official reference. More seriously, the IEEE
standard is parameterised, and AMC had queried whether we had tied them all
down, and specified byte and bit order completely. DPC thought that we had,
but this issue does require checking. A worked example would be helpful.
\item MK would like to extend OpenMath with a genuine Record construct, so
that a Group could be defined as a record of (carrier,operation). He had a
worked suggestion: \verb+<OMR>+ to start a record, individual children
being \verb+OMAVP+ pairs, and \verb+<OMSEL>+ as a selector. DPC pointed out
that this could also be achieved by adding a \verb+record+ CD, which would
not require a language change and not break all existing software. SB
pointed out that, for fixed records, nothing would be needed. The feeling
of the meeting (explicitly checked by JHD) was that the most that should be
done was a \verb+record+ CD.
\item The OpenMath standard essentially predates XML namespaces.
At the very least, we should automatically add
the OpenMath name space properly, but this change should be made as soon as
possible. PL added that OMDoc had already done this.
Could the XML name space be used to make OpenMath less verbose, so that, with
the appropriate declarations, \verb+<alg1:times/>+ could replace
\verb+<OMS name="times" cd="alg1"/>+. Alternatively, we could use
\verb+<OMS name="alg1:times"/>+. JHD argued against the suggestion of
\verb+<OMS name="times" cd="mathml:alg1"/>+, and DPC agreed. Using name
spaces for CDs would mean that globally-unique names for CDs were no longer
needed.  DPC was worried that we didn't have a plausible mechanism for
allocating CD names.
\end{itemize}
In answer to a question from JHD, DPC thought that there was no longer an
8+3 restriction on CD names. He was asked to check this.

%OM-AGM \subsection{Special Meeting of the OpenMath Society}
%OM-AGM AMC called the meeting to order.
%OM-AGM \begin{enumerate}
%OM-AGM \item Present were Buswell, Caprotti, Carlisle,
%OM-AGM Cohen, Davenport, Dewar, Ga\"etano, Naylor, Sander and Seppala. AMC
%OM-AGM reminded the meeting that others could apply if they had attended three
%OM-AGM meetings, or worked for at least three months full-time on OpenMath. The
%OM-AGM Steering Committee comprised Braham, Cohen, Dewar, Ga\"etano, Seppala and
%OM-AGM Watt. JHD was asked to take the minutes.
%OM-AGM \item
%OM-AGM The Meeting noted that the formal annual meeting in Tallahassee had
%OM-AGM happened, and postponed all substantive action to this meeting. Seppala
%OM-AGM said that the minutes of this meeting had been circulated. According to
%OM-AGM Finnish law, these minutes had been checked by elected checkers. The
%OM-AGM minutes of the last Special Meeting in St.~Andrews had been circulated to
%OM-AGM those there present, and posted on the Web by MCD.
%OM-AGM \item There is about \$600CAN in the account in London, Ontario, after paying
%OM-AGM various domain name registration fees. It was noticed that incidents like
%OM-AGM this made it important to have money in an account. Seppala explained the
%OM-AGM history of the registration problems, and said that the Steering Committee
%OM-AGM had to review the Society's business procedures.
%OM-AGM \par
%OM-AGM It would be possible for the OpenMath Society to charge for workshops {\it
%OM-AGM it\/} organised, but MCD pointed out that the Thematic Network was not
%OM-AGM entitled to make a profit from workshops it organised. It seemed clear to
%OM-AGM him that this could be resolved.
%OM-AGM \item Seppala reported that he was in the process of moving to Helsinki, so
%OM-AGM that Tallahassee was no longer an appropriate place for the
%OM-AGM \verb+openmath.org+ site. He thought that it would be ideal if NAG would
%OM-AGM host it. MCD said that this would, in fact, make  task easier, and
%OM-AGM therefore volunteered on behalf of NAG to take it over.
%OM-AGM \item AMC spoke to the issue of OpenMath-endorsed software. He mentioned in
%OM-AGM particular OMDoc. Our original goal of helping computer-algebra systems to
%OM-AGM inter-operate was not close to being achieved.
%OM-AGM He also mentioned the Eindhoven open-source libraries, available from
%OM-AGM \verb+http://crystal.win.tue.nl/download+.
%OM-AGM It was noted that the NAOMI libraries were not up-to-date, but did appear
%OM-AGM in search engines. AMC thought that we should ensure that current products
%OM-AGM appeared higher-up the lists. MCD thought it was difficult for the OpenMath
%OM-AGM Society to do formal endorsing. AMC had a list of software that should be
%OM-AGM developed. SB said that this list should be published, and open to input by
%OM-AGM others.
%OM-AGM \item CDs. The Thematic Network has JHD as CD editor. JHD noted carefully
%OM-AGM that this was not the same as ``sole author''. He also noted that the
%OM-AGM reviewing process was not perfect. MCD added that what reviewing had been
%OM-AGM done was valuable, and AMC expressed the Society's thanks to AS for his
%OM-AGM contributions. AMC said that it was hard to review CDs in the abstract, and
%OM-AGM that, for him, they only made sense in the context of implementing a
%OM-AGM phrasebook. AMC volunteered Eindhoven to look at the polynomial CDs. OC
%OM-AGM suggested a call for contributions and reviewers. DPC noted that there were
%OM-AGM various CDs around that had not been submitted, and he would like there to
%OM-AGM be a mechanical submission process. AMC and OC agreed.  
%OM-AGM \item Under any other business, AC raised the question of an OpenMath logo.
%OM-AGM He suggested a key-based one, and volunteered to get some graphics
%OM-AGM designed. Seppala thought that this was a good idea. This offer was
%OM-AGM accepted.
%OM-AGM \par
%OM-AGM Seppala reported that he had given an OpenMath talk in China. There it had
%OM-AGM been said that it was important to have Chinese versions. WAN pointed out
%OM-AGM that Stephen Watt had several Chinese students.
%OM-AGM \par
%OM-AGM AMC reported that the next OpenMath workshop would be at the ICM in Beijing
%OM-AGM in 2002. MCD noted that IMACS-ACA was having a collection of workshops in
%OM-AGM Greece in the summer of 2002. This would require further discussion.
%OM-AGM \par
%OM-AGM AMC declared the meeting closed.
%OM-AGM \end{enumerate}

\section{OpenMath Workshop 28.9.2001}


\subsection{The M@inline Project}
Peter Sander et al.

Initially Peter Sander gave an overview of the project.

The Group had been running for 18 months, 
Multimedia Applications Involving Non Linear Information for
Networked Education.  

Affiliated with I3S (CNRS) \& ESSI (IT \& engineering school, Univ. Nice).

Previous work on JOME in OpenMath project, Industrial partner loses
rights if does nothing in 6 months so soon have full rights to JOME
and make public CVS. 

Courseware: developing introductory course on Java. Used powerpoint c
800 ppoint slides. need something better. Aim to 
develop platform with database of `bits' of course.  Front end extracts
XML to be delivered as lecture notes, slides, web course notes, etc.
2 student projects (XCC, XIND) implement prototypes:

XIND is an  SVG diagram creator (e.g.~ commutative diagrams).

Infrastructure project: lack of information systems at the school.
student registrations on one machine, mark information in excel on instructor's
machine, course notes elsewhere. so developing in house project to
develop coordinating software, not competing with commercial blackboard/webct

Developing content: something better than latex/latex2html is required
to produce more interactive course.

Current projects: OM (RIP), OM TM,
Trial solution (database of ``slices'' of latex courses),
EML (education modeling language),
TICE (technology and infomatics applied to education) something better
than French translations of (e.g.~ MIT) courses, need to develop native
French/European content, E-MIAGE.


Peter mentioned close collaboration with the CAFE group at INRIA,
Mark Ga\"etano then spoke for the INRIA group.

CAFE: Computer Algebra Functional Equations:
mainly algorithms for differential equations. Started 1998 after
SAFIR. affiliated with INRIA and I3S.

The group develops interfaces to Computer Algebra systems, especially for ODE. More
generally  develop software tools for mathematics.

Manual Bronstein, developing algorithms implemented in Aldor (and some maple).
what to do with this code. Aldor is good but not widely available or used.
Aim to provide implementations as web services so end users don't need Aldor
locally. Notes that these are very specialised systems (not like maple etc).

Hope to use OM/MathML and the formula database still being developed,
hope to release as a web service.
A prototype database was developed on OM project, but next release more usable.
hope to implement a reasonable part of A\&S.

Projects: OM (RIP), OM TM, Cathode.

Finally S\'ephane L. spoke on SVG rendering of mathematics.

Previously worked (MSc) on emath (mathematical editor) at INRIA: common
interface to Computer Algebra (maple, Mathematica, etc), then for PhD on OFR: Optical
formula recognition: recognise printed/handwritten
formula. (96-2000) working on interface to Computer Algebra, sending OM to Mathematica.
Joined m@inline in 2000.

Courseware.

static display: XML/XHTML/SVG (MathML2SVG) (SVG, scalable vector
graphics, is a  W3C Recommendation.)

dynamic display : live documents (XIND, Xind is not Dia)


Demonstrated rendering of Content MathML by the mathml2svg tool using
Adobe SVG plugin in the browser. Demonstrated how this is
movable/zoomable/scalable.


FIXIDEA (son of JOME) 
editing structured mathematical documents (including charts graphs etc)
Java using svg canvas for display.

Future work:

Enhance MMLC -> OM. Add support for new operators.

Develop online course with
mathml2svg and/or OM2svg, FIXIDEA, OM Broker.



\subsection{In Search Of The Semantic Spider}
Stephen Buswell spoke on the relationships between OpenMath and the
Semantic Web.

What is the Semantic Web:
Collections of ideas and technologies, web services, 2-way
transactions, user profiles

What is OM:
2.5 XML languages (CD OM Object, OMDOC. Mathbook,...) and
mathematical resources (CD library, phrasebooks, OM aware applications, OM stylesheets).

Semantic Web has origins in classical metadata:

Genealogy, real estate, pornography (MD for parent/child protection).

librarians (card indexes)

screen scraping

search engines.
yahoo (human, top down) google (automatic, bottom up)

e.g.~ ``Apple sacks fuller'': Austrian fruit farmers...

first steps: PICS:\\
grading by 3rd party proprietary tag values.\\
HTML meta element data.\\
RDF\\
Dublin Core (set of standardised metadata labels)\\
RSS (RDF Site Summary)


Metadata in RDF, modelled (equivalently) as a 
labelled directed graph. or  Triple (Property Resource Value).


RDF has classes, subclasses, transitive relations.
Vocabularies constrained by RDFS (RDF Schema restrict values of
properties)

value-set for terms: data dictionary, term inter-relationship: ontology.

example (namespace soup, daml, RDF, RDFS namespace prefixes all intermingled)

RDF processor should be able to skip over non RDF namespaced elements.

RDF is ontology-neutral.

Structure of RDF XML Tree is not the structure of the information described.

Ontologies: next steps.
\begin{itemize}
\item OIL
\item DAML
\item Frameworks for more complicated ontologies

Domain-neutral (cf KR languages)\\
Content not process (cf KQML)\\
type checking for consistency tests\\
higher ontologies for domain specific reasoning.

\end{itemize}

A web services model

UDDI WSDL (and w3c activity XML Description language) SOAP

User Agent Model - CC/PP

CCPP Composite Capabilities/Preferences Profile.

expressed in RDF.
Driven by mobile phones, allows to describe limited capabilities of
client.

CommonKDS Transaction Model,
two way information flow, negotiation to agree on mutual requirements.

OM-SW comparisons:
\begin{itemize}
\item CD --- micro-ontology
\item CD Group --- domain specific ontology
\item openmath.org --- application-specific UDDI server
\end{itemize}


Next steps:
\begin{itemize}
\item restructure CDs in RDF, easier to do FMPs that are properties of
  two symbols.
\item use URI as unique OM CD symbolname identifier
\item Build mathematical services description language over e.g.~ WSDL
\item develop transaction models
\item develop agent models over e.g.~  CCPP
\item openmath.org application specific UDDI
\item visualisation and navigation
\end{itemize}

\subsection{Semantically Encoded Mathematics On The Web}
Paul Libbrecht spoke on the ActiveMath learning environment.

Current solutions:
\begin{itemize}
\item GIFs: latex2html
\item Mathml
\item HTML and Unicode symbols
\item PDF from \TeX
\item applets (e.g.~jdvi applet from Berlin)
\end{itemize}
Changing standards buggy browsers, unpredictable installations.

Using semantic encoding:
\begin{itemize}
\item goal for author: write what you mean
\end{itemize}
requirements
\begin{itemize}
\item standard declaration of symbols
\item separation of presentation and content
\item organisation of content units in a logical way
\item possibility of explanations on formulas in presented content
\item convertibility to multiple targets. (latex html, maple, mupad, \ldots)
\end{itemize}

Using OMDoc, an XML language based on OM, Has items such as definition
proof example assertion. Allows dual encoding: machine and human.
May be thought of as an ontology that is extensible (symbols have definitions).
OMDoc group has developed several XSL stylesheets converting to html/latex.


Adaptive learning:
\begin{itemize}
\item add a user model
\item add a presentation planner
\item add pedagological rules
\item get a content presentation that is adapted to the user
\item experiment with pedagological theories
\end{itemize}

Example: one student just sees example questions, another sees full
training material, depending on their ``profile''.


Active Math has been developed at DFKI and University of  Saarlandes.
It will be licensed open source. A stable version is expected to be
released next year.

Producing Content for OMDoc, involves two types of author:
choosing mathematical systems, and  developing interfaces
(developer), writing content (author).

Writing omdocs is not easy, especially OM part.

The QMath processor is an editing tool that takes a latex-like syntax to OMDoc.
Also hope to develop a  swing based visual editor.

Future work includes:
\begin{itemize}
\item  enhanced presentation planner
\item more configurable`
visual authoring
\item integration into uni-0nline (for example)
\item gadgets: drag-and-drop from content, slide generation, copyright
  display etc.
\item content being developed within BMBF project.
conversion of Analysis Individuell, a statistics course, formal
methods course.
\end{itemize}

In conclusion,
semantic encoding allows authors to forget the dirty details of the
browser,
and allows for re-use of content.

ActiveMath provides the shell:
authors and developers can now now start writing!

\end{document}
