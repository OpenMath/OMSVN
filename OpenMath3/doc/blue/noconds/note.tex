\documentclass[12pt]{article}
\usepackage{a4wide,url}
\usepackage[hyper]{acronyms}
\usepackage{lstomdoc,xmlindex}
\usepackage[show]{ed}
\usepackage[eso-foot,today]{svninfo}
\usepackage{hyperref}
\svnInfo $Id: note.tex 7995 2008-09-06 05:18:41Z kohlhase $
\svnKeyword $HeadURL: https://svn.omdoc.org/repos/omdoc/trunk/doc/blue/mmturis/note.tex $
\lstset{language=OpenMath,basicstyle=\scriptsize}
\def\llquote#1{\ensuremath{\langle\kern-.25em\langle\hbox{\sl{#1}}\rangle\kern-.25em\rangle}}

\title{{\openmath3} without conditions: A Proposal for a MathML3/OM3 Calculus Content Dictionary}
\author{Michael Kohlhase}

\begin{document}
\maketitle
\begin{abstract}
  We propose a new way of encoding binding operators in {\openmath/\mathml} that
  alleviates the need to introduce {\element{condition}} elements into {\openmath}3. We
  evaluate these ideas by providing a content dictionary {\texttt{calculus3}} that is more
  closely aligned with {\mathml2} representation intuitions as a replacement for the
  {\openmath} standard CD {\texttt{calculus1}}.
\end{abstract}

\section{Introduction}
We are currently reworking the {\openmath} content dictionaries from the ``MathML group''
in an attempt to align the {\openmath3} and {\mathml3} languages. One area of contention
is the fact that {\mathml} allows binding constructions where the bound variables are
restricted by ``qualifier elements'', such as {\element{domainofapplication}},
{\element{condition}}, {\element{uplimit}}, {\element{lowlimit}}, {\element{degree}}, and
{\element{momentabout}}.

Another bone of contention is that {\mathml} often expresses functionals using binding
operators over expressions with bound variables (and qualifiers), whereas {\openmath}
tends to apply the functionals themselves to functions represented with the help of the
$\lambda$ operator. Probably the synchronized {\openmath3}/{\mathml3} content dictionaries
should support both styles, since they appeal to different communities of
mathematicians. We propose a new content dictionary
{\texttt{calculus3}}\footnote{{\texttt{calculus2}} already exists as an experimental CD on
  {\texttt{openmath.org}}} that is more closely aligned with {\mathml2} representation
intuitions as a replacement for the {\openmath} standard CD {\texttt{calculus1}}. The new
content dictionary can be found on the {\openmath3} development repository as
{\url{https://svn.openmath.org/OpenMath3/cd/MathML/calculu3.ocd}}.

\section{Derivatives}

{\mathml} interprets derivatives as operators on expressions in one bound variable and
presents as paradigmatic examples:
\begin{center}\lstset{frame=none,numbers=none,lineskip=-.7ex,aboveskip=-.5em,belowskip=-1em,language=MathML}
\begin{tabular}{|p{4.5cm}|p{8cm}|}\hline
  1: $diff_x(x^2)$ & 2: $diff^2_x(x^5)$\\\hline
\begin{lstlisting}
<apply><diff/>
  <bvar><ci>x</ci></bvar>
  <apply>
    <power/>
    <ci>x</ci>
    <cn>2</cn>
  </apply>
</apply>
\end{lstlisting}
&
\begin{lstlisting}[language=MathML]
<apply><diff/>
  <bvar>
    <ci>x</ci>
    <degree><cn>2</cn></degree>
  </bvar>
  <apply><power/><ci>x</ci><cn>5</cn></apply>
</apply>
\end{lstlisting}
\\\hline
\end{tabular}
\end{center}
but also allows differentiation over a function as in 
\begin{lstlisting}[language=MathML]
<apply><eq/><apply><diff/><sin/></apply><cos/></apply>
\end{lstlisting}
In this, we use the {\element{diff}} element as a functional that is applied to the $\sin$
function. For {\openmath} the functional view is primary: the content dictionary
{\texttt{calculus1}} supplies a symbol {\texttt{diff}} that is a functional so that the
latter expression can be directly represented as 
\begin{lstlisting}[language=OpenMath]
<OMA><OMS cd="relation1" name="eq"/>
  <OMA><OMS cd="calculus1" name="diff"/><OMS cd="transc1" name="sin"/></OMA>
  <OMS cd="transc1" name="cos"/>
</OMA>
\end{lstlisting}
For the left hand expression in the table above, we would use the {\texttt{lambda}} symbol
from the {\texttt{fns1}} CD and a special symbol {\texttt{nthdiff}} from the
{\texttt{calculus1}} CD.
\begin{center}\lstset{frame=none,numbers=none,lineskip=-.7ex,aboveskip=-.5em,belowskip=-1em,language=OpenMath}
\begin{tabular}{|p{6.8cm}|p{7.3cm}|}\hline
  1: $diff_x(x^2)$ & 2: $diff^2_x(x^5)$\\\hline
\begin{lstlisting}
<OMA><OMS cd="calculus1" name="diff"/>
  <OMBIND>
    <OMS cd="fns1" name="lambda"/>
    <OMBVAR><OMV name="x"/></OMBVAR>
    <OMA>
      <OMS cd="arith1" name="power"/>
      <OMV name="x"/>
      <OMI>2</OMI>
    </OMA>
  </OMBIND>
</OMA>
\end{lstlisting}
&
\begin{lstlisting}[language=OpenMath]
<OMA><OMS cd="calculus1" name="nthdiff"/>
  <OMI>2</OMI>
  <OMBIND>
    <OMS cd="fns1" name="lambda"/>
    <OMBVAR><OMV name="x"/></OMBVAR>
    <OMA><OMS cd="arith1" name="power"/>
      <OMV name="x"/>
      <OMI>5</OMI>
    </OMA>
  </OMBIND>
</OMA>
\end{lstlisting}
\\\hline
\end{tabular}
\end{center}
While we lose the directly structural correspondence, this is quite natural. But for a
partial derivative like $\frac{d^k}{dx^m dy^n}f(x,y)$ which can be expressed in {\mathml}
by

\begin{lstlisting}[language=MathML]
<apply>
  <partialdiff/>
    <bvar><ci>x</ci><degree><ci>m</ci></degree></bvar>
    <bvar><ci>y</ci><degree><ci>n</ci></degree></bvar>
    <degree><ci>k</ci></degree>
    <apply><ci type="function">f</ci><ci>x</ci><ci>y</ci></apply>
</apply>
\end{lstlisting}
we would obtain the following representation using a partial differentiation operator that
takes a list of degrees and a total degree as an arguments.
\begin{lstlisting}[language=OpenMath]
<OMA>
  <OMS cd="calculus1" name="pdiffdegree"/>
  <OMA><OMS cd="list1" name="list"><OMV name="m"/><OMV name="m"/></OMA>
  <OMV name="k"/>
  <OMBIND>
    <OMS cd="fns1" name="lambda"/>
    <OMBVAR><OMV name="x"/><OMV name="y"/></OMBVAR>
    <OMA><OMV name="f"><OMV name="x"/><OMV name="y"/></OMA>
  </OMBIND>
</OMA>
\end{lstlisting}
Note that we are using a variant {\texttt{pdiffdegree}} of the {\texttt{partialdiff}}
symbol that allows to specify the total degree as an extra argument. We propose to add
this to the {\texttt{calculus1}} CD.

In the proposed {\texttt{calculus2}} CD, we would model the {\texttt{diff}} and
{\texttt{partialdiff}} as binding operator constructors and thereby make use of the fact
that {\openmath} allows the first child of an {\element{OMBIND}} to be an {\element{OMA}},
not just an {\element{OMS}} as is predominantly used. This gives us a much better
structural similarity in the

\begin{center}\lstset{frame=none,numbers=none,lineskip=-.7ex,aboveskip=-.5em,belowskip=-1em,language=MathML}
\begin{tabular}{|l|p{5.5cm}|p{8cm}|}\hline
 &  {\mathml2} & strict c{\mathml3}\\\hline
 1 &
\begin{lstlisting}
<apply><diff/>
  <bvar><ci>x</ci></bvar>
  <apply>
    <power/>
    <ci>x</ci>
    <cn>2</cn>
  </apply>
</apply>
\end{lstlisting}
&
\begin{lstlisting}
<bind><csymbol cd="calculus2">diff</csymbol>
  <bvar><ci>x</ci></bvar>
  <apply>
    <csymbol cd="arith1">power</csymbol>
    <ci>x</ci>
    <cn>2</cn>
  </apply>
</apply>
\end{lstlisting}
\\\hline
2 & 
\begin{lstlisting}[language=MathML]
<apply><diff/>
  <bvar>
    <ci>x</ci>
    <degree><cn>2</cn></degree>
  </bvar>
  <apply>
    <power/>
    <ci>x</ci>
    <cn>5</cn>
  </apply>
</apply>
\end{lstlisting}
&
\begin{lstlisting}[language=MathML]
<bind>
  <apply>
    <nthdiff/>
    <cn>2</cn>
  </apply>
  <bvar><ci>x</ci></bvar>
  <apply>
    <power/>
    <ci>x</ci>
    <cn>5</cn>
  </apply>
</apply>
\end{lstlisting}
\\\hline
\end{tabular}
\end{center}
We have used strict content MathML to highlight the correspondence for the partial
differentiation example we obtain 
\begin{lstlisting}[language=OpenMath]
<OMBIND>
  <OMA><OMS cd="calculus3" name="pdiffdegree"/>
    <OMV name="m"/><OMV name="m"/><OMV name="k"/>
  </OMA>
  <OMBVAR><OMV name="x"/><OMV name="y"/></OMBVAR>
  <OMA><OMV name="f"><OMV name="x"/><OMV name="y"/></OMA>
</OMBIND>
\end{lstlisting}

\section{Integrals}

For integrals, the situation is similar, {\mathml} interprets derivatives as operators on
expressions in one bound variable and presents as paradigmatic examples the following
three expressions, which differ in which ways the bound variables are handled.
\begin{center}\lstset{frame=none,numbers=none,lineskip=-.7ex,aboveskip=-.5em,belowskip=-1em,language=MathML}
\begin{tabular}{|p{5.9cm}|p{4.4cm}|p{4.4cm}|}\hline
  3: $\int_0^af(x) dx$ & 4: $\int_{x\in D}f(x) dx$ & 5: $\int_Df(x)dx$\\\hline
\begin{lstlisting}
<apply>
  <int/>
  <bvar><ci>x</ci></bvar>
  <lowlimit><cn>0</cn></lowlimit>
  <uplimit><ci>a</ci></uplimit>
  <apply><ci>f</ci>
    <ci>x</ci>
  </apply>
</apply>
\end{lstlisting}
&
\begin{lstlisting}[language=MathML]
<apply>
  <int/>
  <bvar><ci>x</ci></bvar>
  <condition>
    <apply><in/>
      <ci>x</ci>
      <ci>D</ci>
    </apply>
  </condition>
  <apply><ci>f</ci>
    <ci>x</ci>
  </apply>
</apply>
\end{lstlisting}
&
\begin{lstlisting}[language=MathML]
<apply>
  <int/>
  <bvar><ci>x</ci></bvar>
  <domainofapplication>
    <ci>D</ci>
  </domainofapplication>
  <apply><ci>f</ci>
    <ci>x</ci>
  </apply>
</apply>
\end{lstlisting}
\\\hline
\end{tabular}
\end{center}
Example 3. uses the {\element{lowlimit}} {\element{uplimit}} qualifiers that specify an
ordered range of integration by allowing the bound variable to range from 0 to
$a$. Example 4. uses a general {\element{condition}} qualifier that allows to place
restrictions on the bound variable (this is possible on any binding operator), and finally
example 5. uses the {\element{domainofapplication}} qualifier element that restricts the
bound variable to range over a set. 

{\mathml2} also allows integration over a function as in
\begin{center}\lstset{frame=none,numbers=none,lineskip=-.7ex,aboveskip=-.5em,belowskip=-1em,language=MathML}
\begin{tabular}{|p{7.7cm}|p{5.3cm}|}\hline
  6: $\int_{[a,b]}\cos$ & 7: $\int\sin=\cos$\\\hline
\begin{lstlisting}
<apply>
  <int/>
  <interval><ci>a</ci><ci>b</ci></interval>
  <cos/>
</apply>
\end{lstlisting}
&
\begin{lstlisting}
<apply>
  <eq/>
  <apply><int/><sin/></apply>
  <cos/>
</apply>
\end{lstlisting}
\\\hline
\end{tabular}
\end{center}

Examples 5. and 6. can be represented in {\openmath} using the {\texttt{defint}} symbol
from the {\texttt{calculus1}} CD using similar representational intuitions as above: we
apply {\texttt{defint}} symbol that takes a set and a function as arguments to the range
of integration provided and construct a function as a $\lambda$-term. Example 7. works
analogously using the {\texttt{int}} symbol for indefinite integration from
{\texttt{calculus1}}.

{\mathml2} claims that {\element{uplimit}} and {\element{lowlimit}} can be reduced to
{\element{domainofapplication}} or {\element{condition}}, but the convention
$\int_a^bf(x)dx=-\int_b^af(x)dx$ shows that this is not directly possible via the claimed
intuition that $\int_a^b$ can be represented by $\int_{[a,b]}$, since either the interval
$[a,b]$ or $[b,a]$ is nonsensical, or we would be forced to come up with a general notion
of ``reversed interval'' only to fix the integration convention. Thus we have to do
something else for example 3. In particular, the {\element{FMP}} representation of the
convention above in the {\texttt{calculus1}} CD is nonsensical and should be eliminated.

For example 4. we are also in trouble, as we have to construct a set from the
{\element{condition}} in order to use {\texttt{int}} or {\texttt{defint}}. This would in
principle be possible using the {\texttt{suchthat}} symbol from {\texttt{set1}}, but we
need a base set for separation here. But this is not given in example 3. and taking the
``universal set'' is set-theoretically problematic.

In this situation we propose to take the representational distinctions in examples 3. -
6. seriously model the definite integrals as distinct binding constructor as we did for
differentiation. Our new {\texttt{calculus3}} content dictionary supplies a symbol
{\texttt{defint}} that takes two real numbers as arguments: the lower and upper limits of
the range of integration, a symbol {\texttt{defintset}} that takes a set as an argument,
and finally a symbol {\texttt{defintcond}} that takes an expression involving the bound
variable as an argument. 

With this, we directly get the following strict content MathML representations (we have
abbreviated {\texttt{calculus3}} to {\texttt{calc3}} to save space):
\begin{center}\lstset{frame=none,numbers=none,lineskip=-.7ex,aboveskip=-.5em,belowskip=-1em,language=MathML}\begin{tabular}{|p{7.5cm}|p{7.8cm}|}\hline
  3: $\int_0^af(x) dx$ & 4: $\int_{x\in D}f(x) dx$ \\\hline
\begin{lstlisting}
<bind>
  <apply>
    <csymbol cd="calc3">defint</csymbol>
    <cn>0</cn><ci>a</ci>
  </apply>
  <bvar><ci>x</ci></bvar>
  <apply><ci>f</ci><ci>x</ci></apply>
</bind>
\end{lstlisting}
&
\begin{lstlisting}
<bind>
  <apply>
    <csymbol cd="calc3">defintcond</csymbol>
    <apply><in/>
      <ci>x</ci>
      <ci>D</ci>
    </apply>
  </apply>
  <bvar><ci>x</ci></bvar>
  <apply><ci>f</ci><ci>x</ci></apply>
</bind>
\end{lstlisting}
\\\hline\hline
 5: $\int_Df(x)dx$ & 6: $\int_{[a,b]}\cos$\\\hline
\begin{lstlisting}
<bind>
  <apply>
    <csymbol cd="calc3>defintset</csymbol>
    <ci>D</ci>
  </apply>
  <bvar><ci>x</ci></bvar>
  <apply><ci>f</ci><ci>x</ci></apply>
</bind>
\end{lstlisting}
& 
\begin{lstlisting}
<apply>
  <csymbol cd="calc1">defint</csymbol>
  <interval><ci>a</ci><ci>b</ci></interval>
  <cos/>
</apply>
\end{lstlisting}
\\\hline
\end{tabular}
\end{center}

Note that we also propose to extend the old {\texttt{calculus1}} content dictionary with a
symbol {\texttt{defintbounds}} that takes two real numbers as arguments: the lower and
upper limits of the range of integration. The symbol {\texttt{defint}} for definite
integration over functions in {\texttt{calculus1}} already takes the role analogous to
{\texttt{{defintset}}} as example 6 shows. 

\section{Conclusions}

We propose a new content dictionary {\texttt{calculus3}} that is more closely aligned with
{\mathml2} representation intuitions as a replacement for the {\openmath} standard CD
{\texttt{calculus1}}.

For differentiation and integration over expressions with bound variables we should use
the {\texttt{calculus3}} symbols, and for differentiation and integration over functions
we should use {\texttt{calculus1}} symbols. This is directly analogous to the situation
between the content dictionaries {\texttt{s\_data1}} and {\texttt{s\_dist1}}, where the
underspecified usage with data sets and random variables in {\mathml} has been specified
into two different content dictionaries in {\openmath}.

Finally, we remark that the use of binding constructors like we use them in the
{\texttt{calculus3}} content dictionary allows us to move the {\mathml}
{\element{condition}} elements into (suitably defined) binding constructors, so that the
core {\openmath} format need not be extended to achieve synchronization with {\mathml}. In
particular note that an analog to the new symbol {\texttt{defintcond}} that takes an
expression involving the bound variable as an argument cannot be added to
{\texttt{calculus1}}, since the bound variable in the condition cannot be accessed outside
the $\lambda$ term that binds it.

The only thing that has be be changed/clarified in the {\openmath3} standard is that scope
of the bound variable (and thus replacement in $\alpha$-renaming) extends to the binding
operator. As we have to clarify alpha renaming for attributions anyway, this seems like
the lesser evil in comparison with extending the {\openmath} format with a condition
element, in particular since existing {\openmath} only uses symbols as binding operators.

% \section{The {\texttt{calculus3}} Content Dictionary}
% \lstinputlisting[language=omCD,nolol]{../../cd/MathML/calculus3.ocd}


\section{Acknowledgements}
This proposal has been greatly influenced by discussions with Florian Rabe in the context
of the development of the {\omdocv{1.6}} notation definitions.

\bibliographystyle{alphahurl} 
\bibliography{kwarc}
\end{document}

% LocalWords:  ns attr xml byctx cd MMT cdbase mmt arith MMTWS mmtget saxon ci
% LocalWords:  Sacerdoti Coen Zacchiroli kwarc domainofapplication uplimit bvar
% LocalWords:  lowlimit momentabout openmath cn eq OMA transc fns nthdiff OMV
% LocalWords:  OMBIND OMBVAR OMI dx dy partialdiff pdiffdegree csymbol af Df
% LocalWords:  defint FMP suchthat functionals defintset defintcond calc
% LocalWords:  defintbounds
