\documentclass[12pt]{article}
\usepackage{a4wide,url}
\usepackage[hyper]{acronyms}
\usepackage{lstomdoc,xmlindex}
\usepackage[show]{ed}
\usepackage[eso-foot,today]{svninfo}
\usepackage{hyperref}
\svnInfo $Id: note.tex 7995 2008-09-06 05:18:41Z kohlhase $
\svnKeyword $HeadURL: https://svn.omdoc.org/repos/omdoc/trunk/doc/blue/mmturis/note.tex $
\lstset{language=OpenMath,basicstyle=\scriptsize}
\def\llquote#1{\ensuremath{\langle\kern-.25em\langle\hbox{\sl{#1}}\rangle\kern-.25em\rangle}}

\title{{\openmath3} without conditions: A Proposal for a MathML3/OM3 Calculus Content Dictionary}
\author{Michael Kohlhase}

\begin{document}
\maketitle
\begin{abstract}
  We propose a new way of encoding binding operators in {\openmath/\mathml} that
  alleviates the need to introduce {\element{condition}} elements into {\openmath}3. We
  evaluate these ideas by providing a content dictionary {\texttt{calculus?}} that is more
  closely alingned with {\mathml2} representation intuitions as a replacement for the
  {\openmath} standard CD {\texttt{calculus1}}.
\end{abstract}

\section{Introduction}
We are currently reworking the {\openmath} content dictionaries from the ``MathML group''
in an attempt to align the {\openmath3} and {\mathml3} languages. One area of contention
is the fact that {\mathml} allows binding constructions where the bound variables are
restricted by ``qualifier elements'', such as {\element{domainofapplication}},
{\element{condition}}, {\element{uplimit}}, {\element{lowlimit}}, {\element{degree}}, and
{\element{momentabout}}.

Another bone of contention is that {\mathml} often expresses functionals using binding
operators over expressions with bound variables (and qualifiers), whereas {\openmath}
tends to apply the functionals themselves to functions represented with the help of the
$\lambda$ operator. Probably the synchronized {\openmath3}/{\mathml3} content dictionaries
should support both styles, since they appeal to different communities of mathematicians.

\section{Acknowledgements}
This proposal has been greatly influenced by discussions with Florian Rabe in the context
of the developement of the {\omdocv{1.6}} notation definitions.

\bibliographystyle{alphahurl} 
\bibliography{kwarc}
\end{document}

% LocalWords:  ns attr xml byctx cd MMT cdbase mmt arith MMTWS mmtget saxon
% LocalWords:  Sacerdoti Coen Zacchiroli kwarc
