\def\cdca{\textsc{DOC}}


\newif\iflemma
\lemmafalse
\def\OMEGA{{$\Omega${\sc mega}}}    
\def\CALCULEMUS{Calculemus}

\def\ednoteshape{\sf}
\newif\ifshowednotes\showednotesfalse
\def\edshownotes{\showednotestrue}
\DeclareOption{show}{\showednotestrue}
\DeclareOption{hide}{\showednotesfalse}
\ProcessOptions

\newcounter{ednote}
\def\ed@foot#1#2#3% text, type, label
{\def\test{#3}\def\empty{}\footnotetext[\value{ednote}]%
{{\sc{#2}\if\test\empty\else\label{ed:#3}[{#3}]\fi:} \ednoteshape #1}}
\def\ed@note#1#2#3% text, type, label
{\addtocounter{ednote}{1}\message{#2!}%
\ifshowednotes%
\footnotemark[\arabic{ednote}]\ed@foot{#1}{#2}{#3}%
\marginpar{#2(\arabic{ednote})}%
\fi}
\newcommand{\ednote}[2][]{\ed@note{#2}{EdNote}{#1}}
\newcommand{\issue}[2][]{\ed@note{#2}{Issue}{#1}}
\newenvironment{newpart}[1]% text
{\addtocounter{ednote}{1}\edef\new@number{\theednote}\message{New Part!\new@number}
\ifshowednotes\ed@foot{#1}{New Part}{}\marginpar{BegNP(\new@number)}\fi}
{\ifshowednotes\marginpar{EndNP(\new@number)}\fi}

\def\ednotemessage{\ifnum\value{ednote}>0\typeout{}%
\typeout{There are still \arabic{ednote} EdNotes and Issues to resolve!}%
\typeout{}\fi}

%% a bit wider
%\advance\evensidemargin-.8in
%\advance\oddsidemargin-.8in
%\advance\textwidth1.6in
% Vertical space mods from Mike's old 2.09 version.
% Separate paragraphs, and subvert \item to leave less space!
%
\parskip=0.5\baselineskip
\parindent=0pt


\documentclass[draft]{artikel3}
\usepackage{paralist,enumerate,url,fullpage}

\begin{document}

\title{Realizing the Potential of Semantic  Markup for Mathematics\\
  \normalfont\large Future and Emerging Technologies Open Coordination
  Action}
\date{Date of Preparation: \today} 
\author{\normalfont\large  Coordinator: Professor Michael Kohlhase\\
  \parbox[t]{.5\textwidth}{
    \normalfont\normalsize School of Engineering \& Sciences\\
    \normalfont\normalsize International University Bremen, \\
    \normalfont\normalsize Campus Ring 12, D-28759 Bremen, Germany}
\hfill
 \parbox[t]{.5\textwidth}{\raggedleft
  \normalfont\normalsize \texttt{m.kohlhase@iu-bremen.de}\\
\normalfont\normalsize tel: +49 421 200 3140\\
\normalfont\normalsize fax: +49 421 200 3103}
}


\titlepage 

\maketitle
\showednotestrue

\section{Objectives of the Action}

In the 21st Century one of the great challenges facing the scientific
community will be to manage increasingly large bodies of
digitally-encoded information.  To allow this information to be
\emph{re-used} in different contexts, and to be \emph{processed} by
software systems,
%amc: replaced assistants by systems to avoid too much of our own idiom
it will need to be encoded using appropriate
semantic markup.  As a result there has in recent years been an
explosion of interest in the semantic web and related technologies.

In the OpenMath Project and the Thematic Network that followed it, the
European Union supported the development of OpenMath and MathML, two
complementary standards for content and presentation markup of
mathematics.  The original intention was that these would be used to
exchange mathematical objects between software systems, facilitate the
development of ``active'' texts where the mathematical components
could be extracted, manipulated and checked.  As such, these
technologies have been a great success and are now beginning to be
used in a wide range of areas.  However now we see the potential to
combine our existing markup mechanisms with emerging semantic web
technologies to allow for digital mathematical content to be managed,
searched, indexed, validated etc.~in a mathematically meaningful way.

The two key objectives of this proposal are:
\begin{itemize}
\item To foster and extend the use of OpenMath and MathML within the
  European scientific community.
\item To embed these technologies into the semantic web by providing
  views of them grounded in suitable ontologies.  This will enable
  them to be used by applications based on RDF and OWL.
\end{itemize}


\section{Introduction}

The ability to use computers for performing mathematical tasks is
undoubtedly one of the key factors behind recent progress in science,
engineering, and economics.  New technology such as the computational
grid and the semantic web could provide the infrastructure which will
allow researchers to make even greater leaps forward by enhancing
collaboration and providing instant access to massive computational
resources.  On a smaller scale, web service technology holds out the
promise of better interaction between mathematical applications within
a particular community.

These visions rely on the ability to move data and mathematical
objects between pieces of software, and this in turn requires a
mechanism for describing the meaning of those objects.  This issue was
addressed in the Fifth Framework OpenMath and OpenMath Thematic
Network projects, in cooperation with the Worldwide Web Consortium's
(W3C) Math Working Group.  The result is two complementary
standards~--- OpenMath and MathML~--- which allow mathematical objects
to be transmitted and processed by computers in a mathematically
meaningful way, as well as rendered in the current generation of web
browsers.

The coordination action focuses on strengthening OpenMath support for
mathematical communication in the scientific, engineering, and
economics communities, and on supporting European involvement in the
newly-chartered W3C Math Interest Group.  The coordination action will
sponsor activities such as

\begin{enumerate}[(i)]
\item the enhancement and strengthening of the community of OpenMath
  users;
\item further evolution of the OpenMath vocabulary;
\item tailoring OpenMath for use as a technology within the semantic
  web;
\item the development of web and grid services which communicate via
  OpenMath;
\item the development of tools for OpenMath developers.
\end{enumerate}


\section{Strengthening the OpenMath Community} 
\label{community}


As a result of the previous projects there is now a broad community of
people using OpenMath in a variety of ways.  Activities in the areas
of publishing, e-learning, web services and conventional mathematical
software are all in progress.  The main aim of this task is to provide
a basic infrastructure in which these groups can collaborate. This
includes providing web-based and hands-on training, supporting mailing
lists and organizing regular workshops.

This task is at the core of the coordination action since it aims at
strengthening the user base of OpenMath by promoting Content
Dictionaries (CDs), by dissemination and training, setting up an editorial
board for CDs, expert consulting, easy to follow and good practice
guidelines. The OpenMath web site at \texttt{http://www.openmath.org}
will collect the results produced during this action's lifetime, and
maintain and enhance the available content as a central resource for
the OpenMath community. 


% I think that it would be good to have a section after the introduction
% describing the existing OpenMath Community.  This could be structured as
% follows:
% - Mathematical Software (NAG, ORCCA/Maple, Axiom, Cocoa ...)
% - Mathematical Publishing (OMDoc, Algebra Interactive, NAG)
% - E-Learning (Eindhoven, LeActiveMath, Helsinki)
% - Semantic Web (MONET, Stilo)
% - Web Services (MONET, ORCCA/Maple)
% Obviously the more groups and the wider the coverage of topics the
% better.  We might want to think about who we could bring "into the fold"
% as well.



\subsection{Mathematical Software}
\label{sec:msw}
% - Mathematical Software (NAG, ORCCA/Maple, Axiom, Cocoa ...)

The original and primary target user group for OpenMath was the
community of mathematical users and developers. OpenMath has been
developed with the intent of supporting easy interfacing of
mathematical software packages with different yet compatible
functionalities.

Examples of computer algebra systems that speak OpenMath include
libraries such as Aldor, general purpose systems such as Maple,
REDUCE, and AXIOM, special purpose systems such as GAP and CoCoA.

%what to do to strenghten this community 
The coordination action intends to strengthen this community by adding
more visibility to the OpenMath compliant software via the web page,
by collecting and disseminating guidelines for implementors, by active
support for the upgrade to OpenMath version 2.0, by the development
and exchange of Content Dictionaries and Phrasebooks.

\subsection{Mathematical Publishing}
\label{sec:mpubl}
% - Mathematical Publishing (OMDoc, Algebra Interactive, NAG)

The mathematical publishing community broadly includes publishing
companies of scientific literature, software companies distributing
documentation and training material, down to the researchers that
distribute results and lectures on-line.

Publishers of mathematical books are interested in the inclusion of
OpenMath in electronic versions of texts because it enhances
interactivity: with a mouse click, a mathematical object can be
transferred to a computational program or to visualization software.
The same reason holds true for OpenMath to be incorporated in other
publishing activities: the mathematics in published material becomes
truly interactive whether it is demonstrating usage of third-party
algorithms or giving hands-on examples to students.

The publishing community can follow the examples set by Algebra
Interactive~\cite{ida}, by OMDoc activities\cite{OMDOC} and by NAG
(hyper-textual documentation?).

%what to do to strenghten this community
The coordination action intends to strengthen this community by
cooperating with activities such as Nist's Digital Library of
Mathematical
Functions\footnote{\url{http://www.openmath.org/cocoon/openmath//meetings/helsinki2004/Miller/slides.pdf}},
collect tools that simplify inclusion of OpenMath in published
material, include OpenMath in accepted formats such as Docbook.


\subsection{E-Learning}
\label{sec:e-learn}
% - E-Learning (Eindhoven, LeActiveMath, Helsinki)
Novel usage of OpenMath comes from the e-learning community. To
e-learning developers, OpenMath offers the advantage of being a rather
simple language for the representation of mathematics that
standardizes the various notations used in computational software
systems. Authors can produce electronic lectures in a format that can
be visualized on the screen, read out automatically or printed. The
same source may be used to produce handouts of specific lesson-trails
or dynamically adapted to the user profile. The mathematical
expressions can be directly fed to a variety of software packages and
to mathematical web services.

***MK: add the OMDOC usage at CMU? ****

LeActiveMath (IST-507826) is a EU project targeting the development of
a third generation e-learning system that is Language-Enhanced, User
Adaptive, and provides Interactive eLearning for Mathematics. OpenMath
is the internal XML format chosen for the representation of the
mathematical content.  The Helsinki Learning System provides tools to
create multilingual adaptive exercise databases that support learning
in any discipline, currently it covers a typical single variable
calculus course in mathematics. The prospective EU project WebAlt
intends to extend the system with OpenMath support.

% WebALT is a European project 

%what to do to strenghten this community
The coordination action intends to strengthen this community by
increasing dissemination in relevant conferences (IMACS-ACA,
technology conference in mathematics), by producing easy-to-follow
self-contained examples beyond the mathematical field, by enlarging
the number of authors to scientific and engineering fields.


\subsection{Markup Languages}
\label{sec:ml}
% -  (MathML, CML)

The user community of markup languages is growing every day and has
the potential of becoming a large user group for OpenMath, should it
be adopted as the markup language of mathematics inside scientific,
engineering and commercial markups (see the recent workshop
\url{http://scimarkuplang.comm.nsdlib.org/cgi-bin/wiki.pl}).

MathML is a format for describing both the presentation and content of
mathematical objects, and is designed to support OpenMath as a content
descriptor.  The W3C Math Working group was recently dissolved, after
completing an editorial revision of MathML.  A new Math Interest Group
has been set up by W3C to support users of MathML, and an important
aspect of this task will be to support European involvement in it.
One of the members of this Consortium, NAG Ltd, provides one of the
co-chairs of the Math Interest Group.

CML, chemical markup language and STMML, scientific, technical and
medical a markup language, RoboML, Robotic Markup Language, are at
present a few examples of markup languages that only support OpenMath
indirectly through MathML.  Public awareness will increase by crafting
examples in these languages which are representable in MathML only by
resorting to OpenMath Content Dictionaries.




\section{Evolving the OpenMath Vocabulary}
\label{sec:OCD}

The mathematical knowledge is provided in OpenMath by means of
symbols, defined in ``OpenMath Content Dictionaries'' (CDs).  OpenMath
Content Dictionaries make available mathematical definitions and
symbols in an extensible, machine-readable way.

For OpenMath to become widespread it is important to have a critical
mass of CDs. This task is centered on identifying and producing CDs
for those areas of mathematics which are not yet supported by either
MathML or OpenMath. The mathematical content will be represented in
CDs and these CDs will be implemented and tested in one or more
software packages as to demonstrate their usefulness.

Special attention will be devoted to CDs linked to 
\begin{itemize}
  
\item computational software packages which decides to become part of
  the OpenMath community by for instance implementing an OpenMath
  \texttt{export} facility (Maple, Axiom, Singular, CoCoA, Macaulay2,
  Cinderella, Wiris, FORM, \ldots)
  
\item projects that use and/or extend OpenMath (LeActiveMath, MONET,
  Algebra Interactive, \ldots)
  
\item computationally relevant fields ranging from NIST digital
  library of mathematical functions to specific case studies such as
  the problem of matrix-group recognition (solvable by GAP and Magma)
  or dynamic geometry (handled by Cinderella and Geometric Discovery
(see \verb`http://193.146.36.49/discovery`). Special
  effort will be devoted to producing CDs for fields new to OpenMath
  such as financial mathematics, physical and engineering
  applications.

\end{itemize}






\section{OpenMath inside the Semantic Web}

W3C is developing tools that will support the next generation
\emph{semantic} web where information is encoded in such a way that
its meaning is unambiguous and where software agents can process and
reason with it on behalf of human users.  In the semantic web vision,
a piece of mathematics is identified with its meaning, the content,
which must be uniquely understood and unambiguous.  OpenMath has
always focused on the problem of creating a representation of
mathematics which is notation independent and carries the mathematical
content, even before the World Wide Web existed in the way we
experience it now. With the recent development of technologies, like
OWL and RDF, meant to underpin a semantic web, OpenMath can play an
important role as the ontological framework of mathematics. The MONET
project (IST-2001-34145) produced results which have already started
to build ontologies that bridge OpenMath to RDF and OWL.  In
particular, MONET experimented with using OpenMath inside the Web
Ontology Language so that a software component could make decisions by
reasoning about OpenMath objects.

This task will continue to investigate the integration of OpenMath
into the semantic web.  One aspect of this is to allow OpenMath
objects to be manipulated by semantic web tools, as was done in MONET.
Another is to develop and further refine the concept of Content
Dictionaries so that the information contained within them can be
processed by sophisticated tools such as proof assistants and theorem
provers.

\section{Web and Grid Services}


% Rewritten from MKM proposal

Web Service technology is a fundamental development towards
distributed collaboration on the computational grid (often called
\emph{e-science}).  The Web Service paradigm can be used in the
mathematical context to offer controlled access to literature
information, e.g. journal articles, or to databases, e.g.  integral
tables, or to offer access to computational resources, e.g.
differential equation solvers, theorem provers, statistical packages.
Evidence from the EU MONET project (IST-2001-34145) suggests that true
advertisement and discovery of mathematical services requires building
and maintaining a significant amount of mathematical infrastructure
beyond that offered by the web service technologies used in industry.
The MONET framework uses a layered approach for the mathematical parts
of the service description, the \emph{Mathematical Service Description
  Language} (MSDL) in which the mathematical fragments of the
descriptions heavily rely on OpenMath.  The ultimate goal is for a
user to be able to state requirements specifically (``solve
$\int_0^\pi{}\sin(x)\,dx$'') or generically (``find me an available
service which performs definite integration and costs nothing for me
to use'') and a mathematical broker to find the suitable services that
match the request.

GENSS (Grid-Enabled Numerical and Symbolic Services), a joint project
between the Universities of Bath and Cardiff, addresses the
combination of Grid computing and mathematical Web services, and their
extension to deliver mathematical problem analysis, and the code and
the resources to compute the answers, using a common open agent-based
framework.. The project builds on the work on Mathematical Web
Services done in the MONET project.

%what to do to strenghten this community
The coordination action intends to strengthen the usage of OpenMath in
e-science by building a MONET expert group that developes and
maintains standards for mathematical service descriptions. To foster
collaboration, this task will setup a public registry of mathematical
services along with the necessary Problem and Algorithm Description
Libraries.  The group will monitor the new web service technologies
developed by Worldwide Web Consortium and OASIS to ascertain their
compatibility with the universe of mathematical applications.


\section{Tools Development} 
\label{tools}

This task realizes the need for OpenMath tools with special emphasis
on tools that facilitate the creation and usage of CDs. Activities
that develop editors, stylesheets, validation mechanisms, archiving
and searching facilities fall under this task.

Tools such as the following will receive special attention by the
coordination action: 
\begin{itemize}
\item the OpenMath editor under development within the European
  project LeActiveMath
\item the MBase knowledge base for archiving and searching Content
  Dictionaries symbols, mathematical properties, signatures.


\end{itemize}



\newpage
%\setcounter{tocdepth}{2}\tableofcontents

\section{NOT PART OF THE PROPOSAL:  Keep In Mind}
Co-ordination actions are meant to complement other Framework Program
instruments in contributing toward integrating research at European
level through well-planned networking or coordination activities.

This means, we need to list to which projects we are adding bells and
whistles.

GOAL: This document should become the big-picture visionary paper to
rally the OM crowd around for the next part of the OM initiative.  We
need a \emph{new vision} that can be sold to the OM crowd (target 1)
and funding bodies (target 2).


\section{NOT PART OF THE PROPOSAL: Work Plan}
We define a work plan containing a whole range of medium- to long-term
types of networking or co-ordination activities.
\begin{itemize}
\item Performance of studies, analyzes, benchmarking exercises;

test sets for  solving simple problems phrased in OM


\item exchange and dissemination of information

\item exchange and dissemination of good practices

exchanging phrasebooks from one package to another.

Sharing OCDs.
Test each on 3 applications / software systems

\item exchanges of personnel;

offer young researchers for writing phrasebooks;

(small  budget)


\item organization of conference, seminar and meetings;

OpenMath workshop series, and satellites for ISSAC or similar, and
possibly further out, eg economics.

Combine with MathML?

\item setting up of common information systems;

all the data base, knowledge base repository of OCDs

connections with MKM?

Expand and maintain www.openmath.org at NAG


\item setting up of expert groups;

one of these: makers of CAS systems, involved in phrasebooks for
particular sets of OCDs.

$\surd$ editorial board for OCD submission, editing, validation, etc.

\item definition, organization and management of joint or common
  initiatives.

GAP-MAGMA for matrix recognition project

OpenMath for finance,

$\surd$ for chemistry, 

$\surd$ GRID, 

$\surd$ e-learning (LeAM) 

FORM

planar geometry: Cinderella, Botana, ...



\end{itemize}


\end{document}

