\def\cdca{\textsc{DOC}}


\newif\iflemma
\lemmafalse
\def\OMEGA{{$\Omega${\sc mega}}}    
\def\CALCULEMUS{Calculemus}

\def\ednoteshape{\sf}
\newif\ifshowednotes\showednotesfalse
\def\edshownotes{\showednotestrue}
\DeclareOption{show}{\showednotestrue}
\DeclareOption{hide}{\showednotesfalse}
\ProcessOptions

\newcounter{ednote}
\def\ed@foot#1#2#3% text, type, label
{\def\test{#3}\def\empty{}\footnotetext[\value{ednote}]%
{{\sc{#2}\if\test\empty\else\label{ed:#3}[{#3}]\fi:} \ednoteshape #1}}
\def\ed@note#1#2#3% text, type, label
{\addtocounter{ednote}{1}\message{#2!}%
\ifshowednotes%
\footnotemark[\arabic{ednote}]\ed@foot{#1}{#2}{#3}%
\marginpar{#2(\arabic{ednote})}%
\fi}
\newcommand{\ednote}[2][]{\ed@note{#2}{EdNote}{#1}}
\newcommand{\issue}[2][]{\ed@note{#2}{Issue}{#1}}
\newenvironment{newpart}[1]% text
{\addtocounter{ednote}{1}\edef\new@number{\theednote}\message{New Part!\new@number}
\ifshowednotes\ed@foot{#1}{New Part}{}\marginpar{BegNP(\new@number)}\fi}
{\ifshowednotes\marginpar{EndNP(\new@number)}\fi}

\def\ednotemessage{\ifnum\value{ednote}>0\typeout{}%
\typeout{There are still \arabic{ednote} EdNotes and Issues to resolve!}%
\typeout{}\fi}

\documentclass{euproposal}
\usepackage{paralist}

\begin{document}
\setcounter{part}{2}% part B


\title{Development of OpenMath CDs Coordination Action\\
  Acronym: \cdca\\
  Date of Preparation: \date{\today} \\
  Future and Emerging Technologies Open\\
  Coordination Action}
\author{International University  Bremen\\
  Coordinator: Professor Michael Kohlhase\\
%    School of Engineering \& Sciences\\
%    International University Bremen, \\
%    Campus Ring 12\\
%    D-28759 Bremen,    Germany\\
  email: \texttt{m.kohlhase@iu-bremen.de}\\
  fax: +49 421 200 3103}


\titlepage 
\maketitle

\setcounter{tocdepth}{1}\tableofcontents\newpage
\begin{center}\bf
Mathematical Content Dictionaries\\
MCD
\end{center}
\newpage\setcounter{chapter}{0}

\chapter{Objectives of the Action}


\begin{quote}\sf
% ##### short incisive abstract
\end{quote}

%%%% some prose 

The ability to use computers for performing mathematical tasks has
undoubtely been one of the key factor in the progress witnessed in the
last decade in the scientific, engineering, and economics fields. Even
if it might seem that no further advances can be expected in the
electronic processing of mathematics, this impression is easily
contradicted by considering how emerging web technologies might once
more revolutionize the way we do and use computer mathematics.

As scientists, engineers, economists, and mathematicians become more
and more dependent on computers in their daily activities, the
mathematical tasks they perform grow more sophisticated. Supporting
this trend involves a constant improvement in the software, in the
interfaces, but also in the way mathematics is represented
electronically. Collaborative problem solving, mathematical web
services, and e-learning notes are only a few of the activities that
greatly benefit from a standardized electronic language for
mathematics. 

Mathematical knowledge representation is an old problem dating back to
..., however first the computers, and now the world wide web add
several new challenges to what seemed to be a solved issue.
%%% list the challenges
The shift towards doing mathematics on the Internet has emphasized the
need for a portable, extensible, reusable and shareable format for
expressing mathematical objects.

%% MATHML
This need has been recognized some years ago both by the World Wide
Web Consortium and by the European Community. 

%% OpenMath from MKM proposal
%% 
OpenMath is a standard for representing mathematical formulae,
allowing them to be exchanged between computer programs, stored in
databases, or published on the worldwide web.  While the original
designers were mainly developers of computer algebra systems, it is
now attracting interest from other areas of scientific computation and
from many publishers of electronic documents with a significant
mathematical content. 







%Unfortunately, not all items of mathematical knowledge are simple
%formulae: a typical theorem of calculus might be ``If $(a_n)$ and
%$(b_n)$ are two convergent sequences, then $(a_n+b_n)$ is also a
%convergent sequence.'' This can only be converted into a single
%formula by unpacking all the definitions, which would lead to massive
%growth in the formulae, as well as unreadability, as one tackled more
%complicated enunciations. The OMDoc\footnote{{\tt
%    http:/www.mathweb.org/omdoc}.  The author of OMDoc is in the
%  consortium.}  mechanism is an attempt to extend OpenMath to cover
%these sorts of concepts and contexts. It is clear that more work will
%need to be done on Knowledge Representation.


The coordination action focuses on four main goals aimed at
strenghtening the OpenMath support for usage in scientific and
engineering community. Mathematical knowledge is made available in
OpenMath by the mechanism of Content Dictionaries. The coordination
action will sponsor activities such as the further development of the
format of the CDs, the creation of new CDs as required by user
applications, and the development of tools to facilitate using and
making of CDs. These points will receive feedback from the basic
coordination task of building a community of OpenMath users.

\section{CD Format Development}
\label{format}

OpenMath Content Dictionaries are a central mechanism in OpenMath
devised to collect and make available mathematical definitions and
symbols in an extensible, machine readible way. To a limited extent,
CDs are also machine understood and the definitions, properties, and
theorems can be processed by theorem proving software.  This task
concentrates on refining the formalism used for CDs, already revised
in OpenMath 2.0, and giving exact usage guidelines on how applications
should understand CDs.  

\section{CD Content Development} 
\label{content}

For OpenMath to become widespread it is important to produce a
critical mass of Content Dictionaries. This task is centered on
identifying and producing CDs for those areas of mathematics which are
not yet supported by neither MathML nor OpenMath. The mathematical
content will be represented in CDs and these CDs will be implemented
in one or more software packages as further demonstrate the
effectivity of the approach. 

Special attention will be devoted to CDs linked to 
\begin{itemize}
\item computational software packages which decides to become part of the
OpenMath community by for instance implementing a OpenMath
\texttt{export} facility (Singular,CoCoA, Macaulay2 come to mind)

\item projects that use and/or extend OpenMath (LeActiveMath, MONET
  and related come to mind)
  
\item computationally relevant fields ranging from NIST digital
  library of mathematical functions to the problem of matrix-group
  recognition.
\end{itemize}


\section{CD Tools Development} 
\label{tools}

This task realizes the need for tools to facilitate the creation and
usage of CDs. Activities that develop editors, stylesheets, validation
mechanisms, archiving and searching facilities falls under this task.
In particular, this task will combine the above tools towards the
creation of a the mathematical knowledge base of OpenMath CDs as a .

\section{CD Community Building} 
\label{community}

This task is at the core of the coordination action since it aims at
strenghtening the user base of OpenMath by promoting Content
Dictionaries. Dissemination and training, setting up editorial board
for CDs, ....
\end{document}

