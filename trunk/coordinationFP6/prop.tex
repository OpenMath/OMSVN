\def\cdca{\textsc{DOC}}


\newif\iflemma
\lemmafalse
\def\OMEGA{{$\Omega${\sc mega}}}    
\def\CALCULEMUS{Calculemus}

\def\ednoteshape{\sf}
\newif\ifshowednotes\showednotesfalse
\def\edshownotes{\showednotestrue}
\DeclareOption{show}{\showednotestrue}
\DeclareOption{hide}{\showednotesfalse}
\ProcessOptions

\newcounter{ednote}
\def\ed@foot#1#2#3% text, type, label
{\def\test{#3}\def\empty{}\footnotetext[\value{ednote}]%
{{\sc{#2}\if\test\empty\else\label{ed:#3}[{#3}]\fi:} \ednoteshape #1}}
\def\ed@note#1#2#3% text, type, label
{\addtocounter{ednote}{1}\message{#2!}%
\ifshowednotes%
\footnotemark[\arabic{ednote}]\ed@foot{#1}{#2}{#3}%
\marginpar{#2(\arabic{ednote})}%
\fi}
\newcommand{\ednote}[2][]{\ed@note{#2}{EdNote}{#1}}
\newcommand{\issue}[2][]{\ed@note{#2}{Issue}{#1}}
\newenvironment{newpart}[1]% text
{\addtocounter{ednote}{1}\edef\new@number{\theednote}\message{New Part!\new@number}
\ifshowednotes\ed@foot{#1}{New Part}{}\marginpar{BegNP(\new@number)}\fi}
{\ifshowednotes\marginpar{EndNP(\new@number)}\fi}

\def\ednotemessage{\ifnum\value{ednote}>0\typeout{}%
\typeout{There are still \arabic{ednote} EdNotes and Issues to resolve!}%
\typeout{}\fi}

\documentclass{euproposal}
\usepackage{paralist,enumerate}

\begin{document}
\setcounter{part}{2}% part B


\title{Development of OpenMath CDs Coordination Action\\
  Acronym: \cdca\\
  Date of Preparation: \date{\today} \\
  Future and Emerging Technologies Open\\
  Coordination Action}
\author{International University  Bremen\\
  Coordinator: Professor Michael Kohlhase\\
%    School of Engineering \& Sciences\\
%    International University Bremen, \\
%    Campus Ring 12\\
%    D-28759 Bremen,    Germany\\
  email: \texttt{m.kohlhase@iu-bremen.de}\\
  tel: +49 421 200 3140}
  fax: +49 421 200 3103}


\titlepage 
\maketitle

\setcounter{tocdepth}{1}\tableofcontents\newpage
\begin{center}\bf
Mathematical Content Dictionaries\\
MCD
\end{center}
\newpage\setcounter{chapter}{0}

\chapter{Objectives of the Action}


\begin{quote}\sf
% ##### short incisive abstract
\end{quote}

\section*{TODO}
Co-ordination actions are meant to complement other Framework Programme
instruments in contributing toward integrating reasdearch at european level
through well-planned netwokring or coordination activities.

This means, we need to list to which projects we are adding bells and whistles.

\section{Introduction}

The ability to use computers for performing mathematical tasks is
undoubtedly one of the key factors behind recent progress in science,
engineering, and economics.  New technology such as the computational
grid and the semantic web could provide the infrastructure which will
allow researchers to make even greater leaps forward by enhancing
collaboration and providing instant access to massive computational
resources.  On a smaller scale, web service technology holds out the
promise of better interaction between mathematical applications within
a particular community.

These visions rely on the ability to move data and mathematical objects
between pieces of software, and this in turn requires a mechanism for
describing the meaning of those objects.  This issue was addressed in
the Fifth Framework OpenMath and OpenMath Thematic Network projects, in
cooperation with the Worldwide Web Consortium's (W3C) Math Working
Group.  The result is two complementary standards~--- OpenMath and
MathML~--- which allow mathematical objects to be transmitted and
processed by computers in a mathematically meaningful way, as well as
rendered in the current generation of web browsers.

The coordination action focuses on strenghtening OpenMath support for
mathematical communication in the scientific, engineering, and economics
communities, and on supporting European involvement in the
newly-chartered W3C Math Interest Group.  The coordination action will
sponsor activities such as

\begin{enumerate}[(i)]

\item the enhancement and strengthening of the community of OpenMath
users;

\item further evolution of the OpenMath vocabulary;

\item tailoring OpenMath for use as a technology within the semantic
web;

\item the development of web and grid services which communicate via
OpenMath;

\item the development of tools for OpenMath developers.

\end{enumerate}


\section{Strengthening the OpenMath Community} 
\label{community}

%This task is at the core of the coordination action since it aims at
%strenghtening the user base of OpenMath by promoting Content
%Dictionaries. Dissemination and training, setting up editorial board
%for CDs, ....

As a result of the previous projects there is now a broad community of
people using OpenMath in a variety of ways.  Activities in the areas of
publishing, e-learning, web services and conventional mathematical
software are all in progress.  The main aim of this task is to provide a
basic infrastructure in which these groups can collaborate,  This
includes providing web-based and hands-on training, supporting mailing
lists and organising regular workshops.

The W3C Math Working group was recently dissolved, after completing an
editorial revision of MathML.  MathML is a format for describing both
the presentation and content of mathematical objects, and is designed to
support OpenMath as a content descriptor.  A new Math Interest Group has
been set up by W3C to support users of MathML, and an important aspect
of this task will be to support European involvement in it.  One of the
members of this Consortium, NAG Ltd, provides one of the co-chairs of
the Math Interest Group.

\section{Evolving the OpenMath Vocabulary}
\label{sec:OCD}

The mathematical knowledge is provided in OpenMath by means of symbols,
defined in `OpenMath Content Dictionaries' (CDs).  OpenMath Content
Dictionaries make available mathematical definitions and symbols in an
extensible, machine-readible way.

For OpenMath to become widespread it is important to have a 
critical mass of CDs. This task is centered on
identifying and producing CDs for those areas of mathematics which are
not yet supported by either MathML or OpenMath. The mathematical
content will be represented in CDs and these CDs will be implemented
in one or more software packages to demonstrate their usefulness.

Special attention will be devoted to CDs linked to 
\begin{itemize}

\item computational software packages which decides to become part of the
OpenMath community by for instance implementing a OpenMath
\texttt{export} facility (Maple, Axiom, Singular, CoCoA, Macaulay2, \ldots)

\item projects that use and/or extend OpenMath (LeActiveMath, MONET,
Algebra Interactive, \ldots)
  
\item computationally relevant fields ranging from NIST digital
  library of mathematical functions to the problem of matrix-group
  recognition.

\end{itemize}

\section{OpenMath inside the Semantic Web}

W3C is developing tools that will support the next generation
\emph{semantic} web where information is encoded in such a way that its
meaning is unambiguous and where software agents can process and reason
with it on behalf of human users.  The MONET project experimented with
using OpenMath inside the Web Ontology Language so that a software
component could make decisions by reasoning about OpenMath objects.

This task will continue to investigate the integration of OpenMath into
the semantic web.  One aspect of this is to allow OpenMath objects to be
manipulated by semantic web tools, as was done in MONET.  Another is to
develop the concept of Content Dictionaries so that the information
contained within them can be processed by sophisticated tools such as
proof assistants and theorem provers.

\section{Web and Grid Services}


\section{CD Tools Development} 
\label{tools}

This task realises the need for tools to facilitate the creation and
usage of CDs. Activities that develop editors, stylesheets, validation
mechanisms, archiving and searching facilities fall under this task.



\chapter{Work Plan}
We define a work plan containing a
whole range of medium- to long-term types of networking or co-ordination
activities.
\begin{itemize}
\item
Performance of studies, analyses, benchmarking exercises;

test sets for  solving simple problems phrased in OM


\item
exchange and dissemination of information

\item exchange and dissemination of good practices

exchanging phrasebooks from oine package to another.

Sharing OCDs.
Test each on 3 applications / software systems

\item exchanges of personnel;

offer young researchers for writing phrasebooks;

(small  budget)


\item 
organisation of conference, seminar and meetings;

OpenMath workshop series, and satellites for ISSAC or similar, and possibly
further out, eg economics.

Combine with MathML?

\item 
setting up of common information systems;

all the data base, knowledge base repository of OCDs

connections with MKM?

Expand and maintain www.openmath.org at NAG


\item 
setting up of expert groups;

one of these: makers of CAS systems, involved in phrasebooks for
particular sets of OCDs.

editorial board for OCD submission, editing, validation, etc.

\item 
definition, organisation and management of joint or common initiatives.

GAP-MAGMA for matrix recognition project

OpenMath for finance,

for chemistry,

GRID,

e-learning (LeAM)

FORM

planar geometry: Cinderella, Botana, ...



\end{itemize}


\end{document}

