\newif\iflemma
\lemmafalse
\def\OMEGA{{$\Omega${\sc mega}}}    
\def\CALCULEMUS{Calculemus}

\def\ednoteshape{\sf}
\newif\ifshowednotes\showednotesfalse
\def\edshownotes{\showednotestrue}
\DeclareOption{show}{\showednotestrue}
\DeclareOption{hide}{\showednotesfalse}
\ProcessOptions

\newcounter{ednote}
\def\ed@foot#1#2#3% text, type, label
{\def\test{#3}\def\empty{}\footnotetext[\value{ednote}]%
{{\sc{#2}\if\test\empty\else\label{ed:#3}[{#3}]\fi:} \ednoteshape #1}}
\def\ed@note#1#2#3% text, type, label
{\addtocounter{ednote}{1}\message{#2!}%
\ifshowednotes%
\footnotemark[\arabic{ednote}]\ed@foot{#1}{#2}{#3}%
\marginpar{#2(\arabic{ednote})}%
\fi}
\newcommand{\ednote}[2][]{\ed@note{#2}{EdNote}{#1}}
\newcommand{\issue}[2][]{\ed@note{#2}{Issue}{#1}}
\newenvironment{newpart}[1]% text
{\addtocounter{ednote}{1}\edef\new@number{\theednote}\message{New Part!\new@number}
\ifshowednotes\ed@foot{#1}{New Part}{}\marginpar{BegNP(\new@number)}\fi}
{\ifshowednotes\marginpar{EndNP(\new@number)}\fi}

\def\ednotemessage{\ifnum\value{ednote}>0\typeout{}%
\typeout{There are still \arabic{ednote} EdNotes and Issues to resolve!}%
\typeout{}\fi}

\documentclass{euproposal}
\usepackage{paralist}

\begin{document}
\setcounter{part}{2}% part B


\title{Mathematical Content Dictionaries Coordination Action\\
  Acronym: MCD\\
  Date of Preparation: 17 June 2004\\
  Future and Emerging Technologies: FET Open\\
  Coordination Action}
\author{University of Bremen\\
  Coordinator: Professor Michael Kohlhase\\
  \\
  email: \texttt{}\\
  fax: } \titlepage \maketitle

\setcounter{tocdepth}{1}\tableofcontents\newpage
\begin{center}\bf
Mathematical Content Dictionaries\\
MCD
\end{center}
\newpage\setcounter{chapter}{0}

\chapter{Objectives of the Action}


\begin{quote}\sf
% ##### short incisive abstract
\end{quote}

%%%% some prose 

The ability to use computers for performing mathematical tasks has
undoubtely been one of the key factor in the progress witnessed in the
last decade in the scientific, engineering, and economics fields. Even
if it might seem that no further advances can be expected in the
electronic processing of mathematics, this impression is easily
contradicted by considering how emerging web technologies might once
more revolutionize the way we do and use computer mathematics.

As scientists, engineers, economists, and mathematicians become more
and more dependant on computers in their daily activities, the
mathematical tasks they perform grow more sophisticated. Supporting
this trend involves a constant improvement in the software, in the
interfaces, but mostly, in the way mathematics is represented
electronically. 

Mathematical knowledge representation is an old problem dating back to
..., however first the computers, and now the world wide web add
several new challenges to what seemed to be a solved issue.
%%% list the challenges

The shift towards doing mathematics on the Internet has emphasized the
need for a portable, extensible, reusable and shareable format for
expressing mathematical objects. Collaborative problem solving,
mathematical web services, and e-learning notes are only a few of the
activities that greatly benefit from a standardized electronic
language for mathematics.

%% MATHML
This need has been recognized some years ago both by the World Wide
Web Consortium and by the European Community. The W3C promoted  geared exclusively to Math and resulted in the MathML
language.

%% OpenMath from MKM proposal
%% 
OpenMath is a standard for representing mathematical formulae,
allowing them to be exchanged between computer programs, stored in
databases, or published on the worldwide web.  While the original
designers were mainly developers of computer algebra systems, it is
now attracting interest from other areas of scientific computation and
from many publishers of electronic documents with a significant
mathematical content. 







%Unfortunately, not all items of mathematical knowledge are simple
%formulae: a typical theorem of calculus might be ``If $(a_n)$ and
%$(b_n)$ are two convergent sequences, then $(a_n+b_n)$ is also a
%convergent sequence.'' This can only be converted into a single
%formula by unpacking all the definitions, which would lead to massive
%growth in the formulae, as well as unreadability, as one tackled more
%complicated enunciations. The OMDoc\footnote{{\tt
%    http:/www.mathweb.org/omdoc}.  The author of OMDoc is in the
%  consortium.}  mechanism is an attempt to extend OpenMath to cover
%these sorts of concepts and contexts. It is clear that more work will
%need to be done on Knowledge Representation.

%Ever since the pioneering AUTOMATH\footnote{de Bruijn,N.G., The mathematical
%  language AUTOMATH, its usage and some of its extensions. In: {\it Proc. Symp.
%    Automated Deduction\/}, Springer Lecture Notes in Mathematics 125,
%  Springer-Verlag, 1968.} project, there have been significant projects aimed at
%automating or semi-automating mathematical knowledge and proof.  Some of this has
%led to theorem-provers, such as COQ, Isabelle, THEOREMA, VSE and Omega, others to
%mathematical representations such as MIZAR\footnote{{\tt http://mizar.org}.}, also
%represented in the consortium.

\section{CD Format Development}
\section{CD Content Development}
\section{CD Tools  Development}
\section{CD Community Building}
\end{document}
