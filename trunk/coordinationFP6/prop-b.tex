\documentclass{euproposal}
\usepackage{paralist,enumerate,url}
\usepackage[show]{ed}

\def\omcoa{{\sc{OMCoA}}}
\begin{document}
\setcounter{part}{2}% part B

\title{Realizing the Potential of Semantic  Markup for Mathematics\\
  \normalfont\large Future and Emerging Technologies Open Coordination
  Action}

\date{Date of Preparation: \today}
\author{Coordinator: Professor Michael Kohlhase\\
  School of Engineering \& Sciences\\
  International University Bremen, \\
  Campus Ring 12, D-28759 Bremen, Germany\\ 
\texttt{m.kohlhase@iu-bremen.de}\\
tel: +49 421 200 3140 \\ fax: +49 421 200 3103} 

\titlepage 
\maketitle
\newpage
\tableofcontents

\setcounter{chapter}{0}

\section{Proposal Summary}

\textbf{Realizing the Potential of Semantic Markup for Mathematics}\\
\textsc{Acronym:} {\omcoa}\ednote{this is only temporary, we have a macro, so we
  can change easily} \\


\subsection{Strategic objectives}

The two key objectives of this proposal are:
\begin{itemize}
\item To foster and extend the use of OpenMath and MathML within the
  European scientific community.
\item To embed these technologies into the semantic web by providing
  views of them grounded in suitable ontologies.  This will enable
  them to be used by applications based on RDF and OWL.
\end{itemize}


\subsection{Abstract}

There is a critical mass of activity developing around the area of
semantic markup for mathematics embracing the fields of publishing,
e-science, and mathematical software.  Much of this activity is or has
been sponsored by the EU (OpenMath, OpenMathTN, MONET, WebALT,
LeActivemath, MOWGLI, Calculemus, ...), some is sponsored by national
agencies (MathBroker, GENSS, ...), some by multi-national consortia
(W3C's MathML Interest Group, The OpenMath Society, ...), and some by
private enterprise (NAG Ltd., Springer-Verlag, Design Science Inc,
Stilo Technology Ltd. ...).  Coordination of these activities and the
maintenance of agreed standards will have tremendous long-term
benefits for the scientific and technical community in Europe and the
wider world.




\chapter{Objectives}\label{cha:object}

In the 21st Century one of the great challenges facing the scientific
community will be to manage increasingly large bodies of
digitally-encoded information.  To allow this information to be
\emph{re-used} in different contexts, and to be \emph{processed} by
software systems,
%amc: replaced assistants by systems to avoid too much of our own idiom
it will need to be encoded using appropriate semantic markup.  As a
result there has in recent years been an explosion of interest in the
semantic web and related technologies.

In the OpenMath Project~\cite{OM-EU} and the Thematic
Network~\cite{OM-TN} that followed it, the European Union supported
the development of OpenMath and MathML~\cite{}, two complementary standards
for content and presentation markup of mathematics.  The original
intention was that these would be used to exchange mathematical
objects between software systems, facilitate the development of
``active'' texts where the mathematical components could be extracted,
manipulated and checked.  As such, these technologies have been a
great success and are now beginning to be used in a wide range of
areas.  However now we see the potential to combine our existing
markup mechanisms with emerging semantic web technologies to allow for
digital mathematical content to be managed, searched, indexed,
validated etc.~in a mathematically meaningful way.

The two key objectives of this proposal are:
\begin{itemize}
\item To foster and extend the use of OpenMath and MathML within the
  European scientific community.
\item To embed these technologies into the semantic web by providing
  views of them grounded in suitable ontologies.  This will enable
  them to be used by applications based on RDF and OWL.
\end{itemize}


\chapter{Relevance to the objectives of FET Open}
\label{cha:relevance}

Several research groups in Europe are currently using and developing
semantic markup for mathematics. Their activities range from
publishing, to e-science on the grid, and mathematical software
development. In the past, EU-sponsored projects in this area,
(OpenMath, OpenMathTN, MONET, MOWGLI, Calculemus) have developed new
using emerging web technologies and standards for the electronic
communication of mathematics.  Current EU projects (WebALT,
LeActivemath), national agencies (MathBroker, GENSS, ...),
multi-national consortia (W3C's MathML Interest Group, The OpenMath
Society, ...), and some private enterprise (NAG Ltd., Springer-Verlag,
Design Science Inc, Stilo Technology Ltd., Math4More ...) build upon
and exploit such technologies. Integration of all these related
activities is crucial for the advancement in the emerging area of
semantic web technologies. This coordination action will monitor and
align the on-going international efforts that are extending OpenMath
in emerging markup languages for their activity to promote a common
standard.





\chapter{Potential Impact}\label{cha:impact}

Coordination of the wide range of activities that involve semantic markup of mathematics,
the maintenance of agreed standards, the establishments of expert groups and usage
guidelines will have tremendous long-term benefits for the scientific and technical
community in Europe and the wider world.


\chapter{Consortium and Project Resources}\label{cha:consortium}

\begin{description}
\item[International University of Bremen (Coordinator)] Bremen is the lead developer of
  the OMDoc system for representing complete mathematical documents in a
  machine-processable form.  They are involved in the W3C Math Working group and were
  involved in the development of OpenMath~2.
  
\item[NAG Ltd] NAG is Europe's largest vendor of mathematical software.  It is involved
  heavily in the W3C Math Working Group, providing one of the current co-chairs.  NAG uses
  semantic markup to tailor documentation for specific user communities, and to generate
  software components from abstract specifications.  In addition to coordinating the
  Thematic Network is ran the website and developed a range of validation tools for
  OpenMath documents and Content Dictionaries.  It also led the development of the
  OpenMath~2 Standard.
  
\item[University of Bath] The Computing Department at the University of Bath has a long
  history of involvement in Computer Algebra, and until recently was part of the School of
  Mathematical Sciences.  It coordinated the development of Content Dictionaries during
  the former projects: OpenMath and OpenMath Thematic Network.
  
\item[Stilo Technology Ltd] Stilo is an SME providing XML-based services in a broad range
  of areas.  It is involved in a number of W3C activities (including MathML) and was
  heavily involved in the development of OpenMath~2.
  
\item[INRIA Sophia-Antipolis] INRIA is one of the three original W3C host institutions and
  the group at Sophia-Antipolis has been involved in the Math Working group from its
  inception.  They also have a long history in the development of mathematical software
  and have been a key developer within the MONET project.
  
\item[Technical University of Eindhoven] The group at Eindhoven has been involved in the
  development of electronic mathematical textbooks, notably \emph{Algebra Interactive} for
  some years, and are using OpenMath to communicate between user interfaces and a range of
  mathematical engines.  They contributed a significant number of Content Dictionaries
  during the OpenMath Thematic Network project and have played a key design and
  development role for the MONET project.
  
\item[University of Nice] The M@INLINE Project at the University of Nice is interested in
  e-learning and the development of tools for delivering interactive courses in
  Mathematics.  They have developed various OpenMath tools including an editor and a
  rendering agent.
  
\item[Konrad-Zuse Zentrum fur Informationstechnik] ZIB is one of the main developers of
  the Reduce Computer Algebra System.

\item[Johannes Kepler University Linz] \ednote{need something new here, was: RISC is one
    of the largest research centers for Computer Algebra in the world.  It played a
    substantial part in the development of the OpenMath~2 Standard and of the MONET
    protocols.}
  
\item[German Research Centre for Artificial Intelligence (DFKI)] Based at the
  DFKI, the ActiveMath project is developing technology for
  e-learning.  They make substantial use of OMDoc, a mechanism for
  representing mathematical documents which is built on OpenMath.
  
\item[University of Helsinki] The University of Helsinki is also
  interested in e-learning and, through its collaboration with Florida
  State University, has developed a substantial body of material for
  teaching undergraduate calculus. Currently, it is co-ordinating the
  newly funded European digital content for the global networks
  project WebALT, Web Advanced Learning Technologies.  The WebALT
  project will create software and sample content for an XML database
  of mathematical problems to be used in undergraduate university
  courses in mathematics.  The technological solutions developed by
  the project will be applicable in general to any K12 mathematics
  instruction.  Mathematics problems will be originally written in
  English and versions in an initial choice of European languages
  including Catalan, Dutch, English, Finnish and Spanish will be
  automatically generated.
  
\item[University of Cologne] Cologne is interested in the theory
  behind semantic markup for mathematics, and provided in-depth and
  thorough critiques of drafts of both MathML~2 (second edition) and
  OpenMath~2.

\item[University of Pisa]
 
\item[University of Genova] 


  
\end{description}


\chapter{Project Management}\label{cha:management}
\begin{newpart}{adapted from the old OpenMath project via the MKM proposal; this
    needs to be discussed} The proposers intend to adapt the
  management systems that have worked successfully in the OpenMath
  project and other networks to which several of them belonged. The
  following key individuals will be involved in the management of the
  network.
\begin{description}
\item[Project Chairman] Michael Kohlhase
\item[Project Manager] Olga Caprotti with help from Ronald Kieschnick
  (IUB Grants Manager)
\item[Site Leaders] Each site will nominate one site leader.
\item[Work Package Leaders] The Project Manager will nominate one for each Work
  Package (with himself as the leader of Work Package 0) from the site that has
  taken lead responsibility for drafting this Work Package. These may be changed
  by the Project Board.
\item[Task Leaders] These are nominated by the Project Manager on the advice of
  the Work Package Leaders.
\end{description}
The first three categories constitute the Supervisory Board, which
will meet at each of the annual plenary meetings.  It will have the
ultimate responsibility for managing the project. Work Package and
Task Leaders, if not already members, may be invited for the
discussion of particular issues, but will not be allowed to vote.
   
Day-to-day management, and budgetary allocation, will be in the hands
of a smaller Management Board: Project Chairman, Project Manager and
representatives of various interest groups.

In practice, and based on experience in other EU projects to which
many of the members have belonged, it is generally possible to manage
such a coordination action very largely by e-mail, though it is
probable that conference calls between the Management Board will be
necessary as well.
  
The Management Board, as advised by the Project Manager, will have the
responsibility for the management of the project within the contract
between the Commission and the Coordinator.
  
If a conflict arises, or is noticed by the Project Manager, he may
attempt to resolve it himself. If he does not, or if this fails, then
he refers the conflict to the Project Chairman. The latter may also
attempt to resolve it himself. If he does not, or if this fails, then
the Project Chairman refers it to the Management Board. He can either
call a special meeting of the Board (with at least two weeks notice
given to all members by fax and e-mail) or wait until the next
scheduled meeting of the Board, depending on the timing and the
importance of the conflict.  At the meeting, each member to the
conflict may state their case. The Board will then vote by secret
ballot. In the event of a tie, the Chairman will have a casting vote.

It is envisaged that, under the Management Board, there will be
Committees responsible for each of
\begin{itemize}
\item Integrating Activities
\item The Joint Programme of Activities --- all Work Package leaders
  would belong to this Committee;
\item Training and Dissemination Activities.
\end{itemize}
However, the full structure of these Committees would have to await
the first meeting of the Supervisory Board to approve the Management
Board's recommendations.
\end{newpart}

\chapter{Workplan}\label{cha:workplan}



\section{Introduction}

The ability to use computers for performing mathematical tasks is
undoubtedly one of the key factors behind recent progress in science,
engineering, and economics.  New technology such as the computational
grid and the semantic web could provide the infrastructure which will
allow researchers to make even greater leaps forward by enhancing
collaboration and providing instant access to massive computational
resources.  On a smaller scale, web service technology holds out the
promise of better interaction between mathematical applications within
a particular community.

These visions rely on the ability to move data and mathematical
objects between pieces of software, and this in turn requires a
mechanism for describing the meaning of those objects.  This issue was
addressed in the Fifth Framework OpenMath and OpenMath Thematic
Network projects, in cooperation with the Worldwide Web Consortium's
(W3C) Math Working Group.  The result is two complementary
standards~--- OpenMath and MathML~--- which allow mathematical objects
to be transmitted and processed by computers in a mathematically
meaningful way, as well as rendered in the current generation of web
browsers.

The coordination action focuses on strengthening OpenMath support for
mathematical communication in the scientific, engineering, and
economics communities, and on supporting European involvement in the
newly-chartered W3C Math Interest Group.  The coordination action will
sponsor activities such as

\begin{enumerate}[(i)]
\item the enhancement and strengthening of the community of OpenMath
  users;
\item further evolution of the OpenMath vocabulary;
\item tailoring OpenMath for use as a technology within the semantic
  web;
\item the development of web and grid services which communicate via
  OpenMath;
\item the development of tools for OpenMath developers.
\end{enumerate}


\subsection{Strengthening the OpenMath Community}\label{community}


As a result of the previous projects there is now a broad community of
people using OpenMath in a variety of ways.  Activities in the areas
of publishing, e-learning, web services and conventional mathematical
software are all in progress.  The main aim of this task is to provide
a basic infrastructure in which these groups can collaborate. This
includes providing web-based and hands-on training, supporting mailing
lists and organizing regular workshops.

This task is at the core of the coordination action since it aims at
strengthening the user base of OpenMath by promoting Content
Dictionaries (CDs), by dissemination and training, setting up an
editorial board for CDs, expert consulting, easy to follow and good
practice guidelines. The OpenMath web site at
\texttt{http://www.openmath.org} will collect the results produced
during this action's lifetime, and maintain and enhance the available
content as a central resource for the OpenMath community.


% I think that it would be good to have a section after the introduction
% describing the existing OpenMath Community.  This could be structured as
% follows:
% - Mathematical Software (NAG, ORCCA/Maple, Axiom, Cocoa ...)
% - Mathematical Publishing (OMDoc, Algebra Interactive, NAG)
% - E-Learning (Eindhoven, LeActiveMath, Helsinki)
% - Semantic Web (MONET, Stilo)
% - Web Services (MONET, ORCCA/Maple)
% Obviously the more groups and the wider the coverage of topics the
% better.  We might want to think about who we could bring "into the fold"
% as well.



\subsubsection{Mathematical Software}\label{sec:msw}
% - Mathematical Software (NAG, ORCCA/Maple, Axiom, Cocoa ...)

The original and primary target user group for OpenMath was the
community of mathematical users and developers. OpenMath has been
developed with the intent of supporting easy interfacing of
mathematical software packages with different yet compatible
functionalities.

Examples of computer algebra systems that speak OpenMath include
libraries such as Aldor, general purpose systems such as Maple,
REDUCE, and AXIOM, special purpose systems such as GAP and CoCoA.
Furthermore, OpenMath has been taken up as an communication standard
by (semi)-automated theorem provers and program verification systems
through the OMDoc format in the OMEGA, TPS, OTTER, and PVS.

%what to do to strengthen this community 
The coordination action intends to strengthen this community by adding
more visibility to the OpenMath compliant software via the web page,
by collecting and disseminating guidelines for implementors, by active
support for the upgrade to OpenMath version 2.0, by the development
and exchange of Content Dictionaries and Phrasebooks.

\subsubsection{Mathematical Publishing}\label{sec:mpubl}
% - Mathematical Publishing (OMDoc, Algebra Interactive, NAG)

The mathematical publishing community broadly includes publishing
companies of scientific literature, software companies distributing
documentation and training material, down to the researchers that
distribute results and lectures on-line.

Publishers of mathematical books are interested in the inclusion of
OpenMath in electronic versions of texts because it enhances
interactivity: with a mouse click, a mathematical object can be
transferred to a computational program or to visualization software.
The same reason holds true for OpenMath to be incorporated in other
publishing activities: the mathematics in published material becomes
truly interactive whether it is demonstrating usage of third-party
algorithms or giving hands-on examples to students.

The publishing community can follow the examples set by Algebra Interactive (TU
Eindhoven/Springer), by OMDoc activities (IUB, DFKI, Carnegie Mellon University) and by
NAG (hyper-textual documentation).\ednote{needs Pisa}

%what to do to strengthen this community
The coordination action intends to strengthen this community by cooperating with
activities such as the Digital Library of Mathematical Functions (DLMF) developed at the
National Institute of Standards and Technology (NIST)
\footnote{\url{http://www.openmath.org/meetings/helsinki2004/Miller/slides.pdf}}, collect
tools that simplify inclusion of OpenMath in published material, include OpenMath in
accepted formats such as DocBook.\ednote{Some text from Pisa on Active Journals}

\subsubsection{E-Learning}\label{sec:e-learn}
% - E-Learning (Eindhoven, LeActiveMath, Helsinki)
Novel usage of OpenMath comes from the e-learning community. To
e-learning developers, OpenMath offers the advantage of being a rather
simple language for the representation of mathematics that
standardizes the various notations used in computational software
systems. Authors can produce electronic lectures in a format that can
be visualized on the screen, read out automatically or printed. The
same source may be used to produce handouts of specific lesson-trails
or dynamically adapted to the user profile. The mathematical
expressions can be directly fed to a variety of software packages and
to mathematical web services.

LeActiveMath (IST-507826) is a EU project targeting the development of
a third generation e-learning system that is Language-Enhanced, User
Adaptive, and provides Interactive eLearning for Mathematics. OpenMath
is the internal XML format chosen for the representation of the
mathematical content.  The Helsinki Learning System provides tools to
create multilingual adaptive exercise databases that support learning
in any discipline, currently it covers a typical single variable
calculus course in mathematics. The prospective EU project WebAlt
intends to extend the system with OpenMath support.

  Outside the EU, the ``Open Learning Initiative'' and the ``Course Capsules
  Project'' at Carnegie Mellon University have built on the OpenMath/OMDoc format,
  extending it to courses in Computer Science and Philosophy, adding authoring
  support in the {\LaTeX}, MS-PowerPoint, and Mathematica workflows.

The coordination action intends to strengthen this community by
increasing dissemination in relevant conferences (IMACS-ACA,
technology conference in mathematics), by producing easy-to-follow
self-contained examples beyond the mathematical field, by enlarging
the number of authors to scientific and engineering fields.

\subsubsection{Markup Languages}\label{sec:ml}

The user community of markup languages is growing every day and has
the potential of becoming a large user group for OpenMath, should it
be adopted as the markup language of mathematics inside scientific,
engineering and commercial markups (see the recent workshop
\url{http://scimarkuplang.comm.nsdlib.org/cgi-bin/wiki.pl}).

MathML is a format for describing both the presentation and content of
mathematical objects, and is designed to support OpenMath as a content
descriptor.  The W3C Math Working group was recently dissolved, after
completing an editorial revision of MathML.  A new Math Interest Group
has been set up by W3C to support users of MathML, and an important
aspect of this task will be to support European involvement in it.
One of the members of this Consortium, NAG Ltd, provides one of the
co-chairs of the Math Interest Group.

CML (Chemical Markup Language) and STMML (Scientific, Technical and
Medical Markup Language), RoboML (Robotic Markup Language), are at
present a few examples of markup languages that only support OpenMath
indirectly through MathML.  Public awareness will increase by crafting
examples in these languages which are representable in MathML only by
resorting to OpenMath Content Dictionaries.

\subsection{Evolving the OpenMath Vocabulary}\label{sec:OCD}

The mathematical knowledge is provided in OpenMath by means of
symbols, defined in ``OpenMath Content Dictionaries'' (CDs).  OpenMath
Content Dictionaries make available mathematical definitions and
symbols in an extensible, machine-readable way.

For OpenMath to become widespread it is important to have a critical
mass of CDs. This task is centered on identifying and producing CDs
for those areas of mathematics which are not yet supported by either
MathML or OpenMath. The mathematical content will be represented in
CDs and these CDs will be implemented and tested in one or more
software packages as to demonstrate their usefulness.

Special attention will be devoted to CDs linked to 
\begin{itemize}
\item computational software packages which decides to become part of
  the OpenMath community by for instance implementing an OpenMath
  \texttt{export} facility (Maple, Axiom, Singular, CoCoA, Macaulay2,
  Cinderella, Wiris, FORM, \ldots)
\item projects that use and/or extend OpenMath (LeActiveMath, MONET,
  Algebra Interactive, \ldots)
\item computationally relevant fields ranging from the Digital Library
  of Mathematical Functions at NIST to specific case studies such as
  the problem of matrix-group recognition (solvable by GAP and Magma)
  or dynamic geometry (handled by Cinderella and Geometric Discovery
  (see {\url{http://193.146.36.49/discovery}}). Special effort will be
  devoted to producing CDs for fields new to OpenMath such as
  financial mathematics, physical and engineering applications.
\end{itemize}
With the OpenMath2 standard accomplished in the OpenMathTN, we have generalized
the content dictionary format, so that content dictionaries e.g. in OMDoc form can
now be self-documenting entities that also develop the necessary mathematical
material. In this way, e.g. the DLMF can serve as a web-accessible tome of
information about special functions as well as their content dictionary, which
fixes the semantics of OpenMath objects referencing them. This removes the need
for separate content dictionaries and alleviates a major knowledge management
hassle and roadblock to content dictionary adaption.

\subsection{OpenMath inside the Semantic Web}

W3C is developing tools that will support the next generation {\emph{semantic}}
web where information is encoded in such a way that its meaning is unambiguous and
where software agents can process and reason with it on behalf of human users.  In
the semantic web vision, a piece of mathematics is identified with its meaning,
the content, which must be uniquely understood and unambiguous.  OpenMath has
always focused on the problem of creating a representation of mathematics which is
notation independent and carries the mathematical content, even before the World
Wide Web existed in the way we experience it now. With the recent development of
technologies, like OWL and RDF, meant to underpin a semantic web, OpenMath can
play an important role as the ontological framework of mathematics. The MONET
project (IST-2001-34145) produced results which have already started to build
ontologies that bridge OpenMath to RDF and OWL.  In particular, MONET experimented
with using OpenMath inside the Web Ontology Language so that a software component
could make decisions by reasoning about OpenMath objects.

This task will continue to investigate the integration of OpenMath
into the semantic web.  One aspect of this is to allow OpenMath
objects to be manipulated by semantic web tools, as was done in MONET.
Another is to develop and further refine the concept of Content
Dictionaries so that the information contained within them can be
processed by sophisticated tools such as proof assistants and theorem
provers.

\subsection{Web and Grid Services}


% Rewritten from MKM proposal

Web Service technology is a fundamental development towards distributed
collaboration on the computational grid (often called {\emph{e-science}}).  The
Web Service paradigm can be used in the mathematical context to offer controlled
access to literature information, e.g. journal articles, or to databases, e.g.
integral tables, or to offer access to computational resources, e.g.  differential
equation solvers, theorem provers, statistical packages.  Evidence from the EU
MONET project (IST-2001-34145) suggests that true advertisement and discovery of
mathematical services requires building and maintaining a significant amount of
mathematical infrastructure beyond that offered by the web service technologies
used in industry.  The MONET framework uses a layered approach for the
mathematical parts of the service description, the \emph{Mathematical Service
  Description Language} (MSDL) in which the mathematical fragments of the
descriptions heavily rely on OpenMath.  The ultimate goal is for a user to be able
to state requirements specifically (``solve $\int_0^\pi{}\sin(x)\,dx$'') or
generically (``find me an available service which performs definite integration
and costs nothing for me to use'') and a mathematical broker to find the suitable
services that match the request.

GENSS (Grid-Enabled Numerical and Symbolic Services), a joint project between the
Universities of Bath and Cardiff, addresses the combination of Grid computing and
mathematical Web services, and their extension to deliver mathematical problem
analysis, and the code and the resources to compute the answers, using a common
open agent-based framework.. The project builds on the work on Mathematical Web
Services done in the MONET project.

%what to do to strengthen this community
The coordination action intends to strengthen the usage of OpenMath in e-science
by building a MONET expert group that develops and maintains standards for
mathematical service descriptions. To foster collaboration, this task will setup a
public registry of mathematical services along with the necessary Problem and
Algorithm Description Libraries.  The group will monitor the new web service
technologies developed by Worldwide Web Consortium and OASIS to ascertain their
compatibility with the universe of mathematical applications.


\subsection{Tools Development}\label{tools}

This task realizes the need for OpenMath tools with special emphasis on tools that
facilitate the creation and usage of CDs. Activities that develop editors,
stylesheets, validation mechanisms, archiving and searching facilities fall under
this task.

Tools such as the following will receive special attention by the coordination
action:
\begin{itemize}
\item the OpenMath/OMDoc editor under development within the European project
  LeActiveMath
\item the MBase knowledge base for archiving and searching Content Dictionaries
  symbols, mathematical properties, signatures.
  \begin{newpart}{MiKo: need more}
  \item The editor CPoint (OpenMath/OMDoc in MS PowerPoint) under development at
    University Bremen
  \item The OpenMath/OMDoc extension for the GNU {\TeX}macs editor under
    development at Saarland University.
  \item The S{\TeX}/{\LaTeX}ML workflow for semantic annotation and conversion of
    {\LaTeX} to OpenMath under development at NIST and IUB.
  \item second-generation presentation systems for OpenMath: in particular modular
    formats for including notation information into content dictionaries and their
    integration into OpenMath presentation- and translation systems.
\end{newpart}
\end{itemize}

\section{Work Plan}

The work plan of the {\omcoa} project is centered around six work packages, one
for each area identified in the previous section (with exception of the tools
development task from~\ref{tools}, which is split into tools development tasks in
all of the work packages) plus one for management. We have not added a separate
work package for dissemination and evaluation, since this is part of each
work-package.

As the work packages are relatively orthogonal, they can basically run in
parallel, and will run for the whole duration of the project.\ednote{adapt and
  extend, where necessary}

The following two tables give an overview over the work packages and their
deliverables.\ednote{please volunteer (evenly) for lead contrator roles. }

\begin{center}
\begin{tabular}{|>{\centering}p{13mm}|p{9cm}|
       >{\centering}p{12mm}|>{\centering}p{8mm}|>{\centering}p{6mm}|}\hline
\bfseries Work\-package No&\bfseries Work package Title&
\bfseries Lead&
\bfseries Start mo.&
\bfseries End mo.\tabularnewline\hline
WP 0& Management                            & IUB  & 1 & 36 \tabularnewline\hline
WP 1& OpenMath in Mathematical Software     & ???   & 1 & 36 \tabularnewline\hline
WP 2& OpenMath in Publishing \& E-Learning  & ???   & 1 & 36 \tabularnewline\hline
WP 3& OpenMath in Markup Languages          & ???   & 1 & 36 \tabularnewline\hline
WP 4& Evolving the OpenMath Vocabulary     & ???   & 1 & 36 \tabularnewline\hline
WP 5& OpenMath and the Semantic Web        & ???   & 1 & 36 \tabularnewline\hline
WP 6& OpenMath in Web- and Grid Services   & NAG?  & 1 & 36 \tabularnewline\hline
WP 7& Maintenance and Development of the OpenMath Standard & ??? & 1 & 36\tabularnewline\hline
\end{tabular}
\end{center}
The next table gives an overview over the deliverables of the work
packages. Note that the times of deliverables after month 18 are
estimations and may change as the work packages progress.

\begin{longtable}{|l|l|p{1cm}|l|l|}\hline
\bfseries Number &
\bfseries Deliverable\slash milestone title&
\bfseries due (mo)&
\bfseries Nature&
\bfseries Dissemination level\tabularnewline\hline\hline
D1.1 & ?????????????           & ?? & ?? & public \tabularnewline\hline
\end{longtable}
\vfill\eject

\section{Work Packages}
\newenvironment{workpackage}[5]% duration, number, start, participants, title
{\begin{center}{\Large\bf Workpackage description\par
   WP {#2}: {#5} ({#1} Months)  \vspace{1cm}}
  \fbox{\parbox[t]{\textwidth-15pt}{\raggedright
\makebox[.4\hsize][l]{\textbf{Workpackage number:} WP{#2}}%
\makebox[.6\hsize][l]{\textbf{Start date or start event:} {#3}}\\
\textbf{Participants:} {#4}}}}
{\end{center}}
\newsavebox{\fmbox}
\newenvironment{fmpage}[1]{\begin{lrbox}{\fmbox}\begin{minipage}[{#1}}{\end{minipage}\end{lrbox}\fbox{\usebox{\fmbox}}}
\newenvironment{wpbox}[1]%
{\begin{lrbox}{\fmbox}\begin{minipage}[t][4cm]{\textwidth-15pt}{\bf{#1}\qquad}}%
{\end{minipage}\end{lrbox}\fbox{\usebox{\fmbox}}}

%%%%%%%% WP 0 
\begin{workpackage}{36}{0}{Start of Project}{IUB}{Management}
\begin{wpbox}{Objectives}
  The aim of this work package is to manage the project, liaise with
  the Commission and the reviewers appointed by them, to respond to
  external events, and to ensure that the aims and objectives of the
  overall project are met.
\end{wpbox}
\begin{wpbox}{Description of Work}
Not applicable.
\end{wpbox}
\begin{wpbox}{Deliverables}
  Not applicable (although it will produce internal guidelines and
  regular management reports as well as the final report, as required
  under the contract).
\end{wpbox}
\begin{wpbox}{Milestones and expected result}
\end{wpbox}
\end{workpackage}
\newpage

%%%%%%%%%%%%%% WP 1
\begin{workpackage}{36}{1}{Project Start}{UNIGE,TUE}{OpenMath in Software}
\begin{wpbox}{Objectives}
  To update the software to be compliant with version 2.0 of the OpenMath standard
  and at the same time build the expertise and guidelines on how to implement
  OpenMath in software. Finally, to build up a body of Content Dictionaries that
  can serve as an interoperability vocabulary for a set of mathematical software
  systems.
\end{wpbox}
\begin{wpbox}{Description of work}
  The work group will convene a workshop in month 6 to asses interoperability
  scenarios, phrasebooks, and vocabulary needs. It will document it's findings in
  a report and will develop a roadmap for a selected field of mathematics, where
  the interoperability via OpenMath is attempted for three or more systems. A
  second developer's Workshop in Month 24 verifies the progress and prepares the
  software release in Month 30.
\end{wpbox}
\begin{wpbox}{Deliverables\\}
  {\bf D1.1} Report on Current Status of OpenMath-compliant Software\\
  {\bf D1.2} Prototype Release of OpenMath2.0-compliant Software\\
  {\bf D1.3} Final OpenMath2.0-compliant Software Upgrade accompanied
  by documentation on the upgrade
\end{wpbox}
\begin{wpbox}{Milestones and expected result}
  The result of this work package will be a set of three or more mathematical
  software packages that interoperate by exchanging OpenMath Objects that derive
  their meaning from a set of commonly agreed Content Dictionaries that are
  implemented by the Phrasebooks of these systems.
\begin{itemize}
\item (Month 6) First Developer's Workshop with subsequent report (D1.1)
\item (Month 24) Second Developer's Workshop with subsequent report (D1.2)
\item (Month 30) Software Release (D1.3)
\end{itemize}
\end{wpbox}
\end{workpackage}
\newpage

%%%%%%%%%%%%%% WP 2
\begin{workpackage}{36}{2}{Project Start}{UNIPI, TUE, IUB}{OpenMath in Publishing
    and E-Learning}
\begin{wpbox}{Objectives}
  OpenMath can play a vital role in electronic publishing and e-learning, as it
  allows to embed the content of mathematical formulae into electronic documents
  and thus allows to enliven mathematical documents. The objective of this work
  package is to coordinate the prototypical efforts in this direction, to develop
  demonstration documents and to establish best-practice guidelines.
\end{wpbox}
\begin{wpbox}{Description of work}
  The work-group will survey the status quo of OpenMath in E-Publishing.  Building on this
  we will organize a workshop with academic publishers and professional societies to
  obtain their feedback and develop best practices guidelines for publishers.
  
  Coordinate the usage of OpenMath in e-learning frameworks by ....exchanging personnel,
  visiting respective project meetings, comparison study?
\end{wpbox}
\begin{wpbox}{Deliverables\\}
{\bf D2.1} Report on OpenMath/Content MathML usage in Publishing\\
{\bf D2.2} Report: Tools and Best Pracices for OpenMath in Electronic Documents
\end{wpbox}
\begin{wpbox}{Milestones and expected result}
\begin{itemize}
\item (Month 6) Status Quo Report (D2.1)
\item (Month 24) Demonstrators for E-Publishing and E-Learning 
\item (Month 36) Best Practices Report (D2.2)
\end{itemize}
\end{wpbox}
\end{workpackage}
\newpage

%%%%%%%%%%%%%% WP 3
\begin{workpackage}{36}{3}{Project Start}{BATH, IUB}{OpenMath in Markup Languages}
\begin{wpbox}{Objectives}
  To encourage the adoption of OpenMath/MathML in markup languages that need to
  represent mathematical objects.  To continue liaison between OpenMath and
  MathML. 
\end{wpbox}
\begin{wpbox}{Description of work}
  The {\omcoa} action will act as a central clearing house and competence center
  for OpenMath. The working group will actively establish contact to all relevant
  markup language development groups and liaise with them. This will largely be
  bi-lateral projects between workgroup members and representatives of these
  groups. 
  
  The group will furthermore experiment with content dictionaries for
  extra-mathematical objects (e.g.  observables in physics) building on the
  experiences with CDs for units.
\end{wpbox}
\begin{wpbox}{Deliverables\\}
{\bf D3.1} \\
{\bf D3.2} \\
{\bf D3.3} \\
\end{wpbox}
\begin{wpbox}{Milestones and expected result}
  The result of this work package will be either the adoption of content-oriented
  representations for Mathematical Objects (OpenMath or Content Mathml) in other
  Markup Languages, or determine the reasons, why this has not taken place. We
  consider the latter unlikely. Any experiences gained in this enterprise will
  result in requirements for further development of the OpenMath Standard (see WP7)
\begin{itemize}
\item (Month 12) ????
\item (Month 24) ???
\item (Month 36) ???
\end{itemize}
\end{wpbox}
\end{workpackage}
\newpage

%%%%%%%%%%%%%% WP 4
\begin{workpackage}{36}{4}{Project Start}{TUE, NAG, IUB, BATH}{Evolving the OpenMath Vocabulary} 
\begin{wpbox}{Objectives}
  To encourage the adoption of OpenMath by providing a large set of OpenMath
  content dictionaries. The development and coordination of tools for archiving,
  finding, and validating CDs.
\end{wpbox}
\begin{wpbox}{Description of work}
  The work group will survey the status quo for OpenMath Content Dictionaries, and
  will convene working groups for developing standard CDs for selected areas. One
  of these will be a set of CDs for computational software packages (with WP1).
  Furthermore, the work group will foster and coordinate the development of an
  Digital Library System for OpenMath Content Dictionaries.
\end{wpbox}
\begin{wpbox}{Deliverables\\}
{\bf D4.1} Report on Status Quo on OpenMath Content Dictionaries\\
{\bf D4.2} Content Dictionaries for Computational Software Packages\\
{\bf D4.3} Final Report on the state of the OpenMath Vocabulary\\
\end{wpbox}
\begin{wpbox}{Milestones and expected result}
\begin{itemize}
\item (Month 3) Assessment of the current state (Deliverable D4.1)
\item (Month 24) First draft for CDs for computational Software Packages
  (Deliverable D4.2)
\item (Month 34) Final Report on the OpenMath Vocabulary (D4.3)
\end{itemize}
\end{wpbox}
\end{workpackage}
\newpage

%%%%%%%%%%%%%% WP 5
\begin{workpackage}{36}{5}{Project Start}{NAG, RISC, BATH}{OpenMath and the Semantic Web}
\begin{wpbox}{Objectives}
  To investigate the integration of OpenMath into the semantic web.  One aspect of
  this is to allow OpenMath objects to be manipulated by semantic web tools.
  Another is to develop and further refine the concept of Content Dictionaries so
  that the information contained within them can be processed by sophisticated
  tools such as proof assistants and theorem provers.
\end{wpbox}
\begin{wpbox}{Description of work}
  ????????
\end{wpbox}
\begin{wpbox}{Deliverables\\}
{\bf D5.1} \\
{\bf D5.2} \\
{\bf D5.3} 
\end{wpbox}
\begin{wpbox}{Milestones and expected result}
\begin{itemize}
\item (Month 12) ????
\item (Month 24) ???
\item (Month 36) ???
\end{itemize}
\end{wpbox}
\end{workpackage}
\newpage

%%%%%%%%%%%%%% WP 6
\begin{workpackage}{36}{6}{Project Start}{RISC, BATH}{Mathematical Services in e-Science}
\begin{wpbox}{Objectives}
  ???????
\end{wpbox}
\begin{wpbox}{Description of work}
  ????????
\end{wpbox}
\begin{wpbox}{Deliverables\\}
{\bf D6.1} \\
{\bf D6.2} \\
{\bf D6.3} 
\end{wpbox}
\begin{wpbox}{Milestones and expected result}
\begin{itemize}
\item (Month 12) ????
\item (Month 24) ???
\item (Month 36) ???
\end{itemize}
\end{wpbox}
\end{workpackage}

\newpage

%%%%%%%%%%%%%% WP 7
\begin{workpackage}{36}{7}{Project Start}{NAG, BATH, IUB}{Maintenance and Development
    of the OpenMath Standard}
\begin{wpbox}{Objectives}
  To maintain the OpenMath 2.0 standard: collect errata, answer questions, write
  interpreting documents. To incorporate new requirements from the project work
  packages into a new draft standard for OpenMath if necessary.
\end{wpbox}
\begin{wpbox}{Description of work}
  The working group will  ?????
\end{wpbox}
\begin{wpbox}{Deliverables\\}
{\bf D7.1} The OpenMath 2.0 primer\\
{\bf D7.2} OpenMath Compliance Guidelines\\
\end{wpbox}
\begin{wpbox}{Milestones and expected result}
\begin{itemize}
\item (Month 12) The OpenMath Primer (D7.1)
\item (Month 24) 
\item (Month 36) 
\end{itemize}
\end{wpbox}
\end{workpackage}

\newpage
%\setcounter{tocdepth}{2}\tableofcontents

\section{NOT PART OF THE PROPOSAL:  Keep In Mind}
Co-ordination actions are meant to complement other Framework Program
instruments in contributing toward integrating research at European
level through well-planned networking or coordination activities.

This means, we need to list to which projects we are adding bells and
whistles.

GOAL: This document should become the big-picture visionary paper to
rally the OM crowd around for the next part of the OM initiative.  We
need a \emph{new vision} that can be sold to the OM crowd (target 1)
and funding bodies (target 2).


\section{NOT PART OF THE PROPOSAL: Work Plan}
We define a work plan containing a whole range of medium- to long-term
types of networking or co-ordination activities.
\begin{itemize}
\item Performance of studies, analyzes, benchmarking exercises;

test sets for  solving simple problems phrased in OM


\item exchange and dissemination of information

\item exchange and dissemination of good practices

exchanging phrasebooks from one package to another.

Sharing OCDs.
Test each on 3 applications / software systems

\item exchanges of personnel;

offer young researchers for writing phrasebooks;

(small  budget)


\item organization of conference, seminar and meetings;

OpenMath workshop series, and satellites for ISSAC or similar, and
possibly further out, eg economics.

Combine with MathML?

\item setting up of common information systems;

all the data base, knowledge base repository of OCDs

connections with MKM?

Expand and maintain www.openmath.org at NAG


\item setting up of expert groups;

one of these: makers of CAS systems, involved in phrasebooks for
particular sets of OCDs.

$\surd$ editorial board for OCD submission, editing, validation, etc.

\item definition, organization and management of joint or common
  initiatives.

GAP-MAGMA for matrix recognition project

OpenMath for finance,

$\surd$ for chemistry, 

$\surd$ GRID, 

$\surd$ e-learning (LeAM) 

FORM

planar geometry: Cinderella, Botana, ...
\end{itemize}
\end{document}


% LocalWords:  mega Calculemus ednote EdNote newpart BegNP EndNP EdNotes pt tel
% LocalWords:  OpenMathTN WebALT LeActivemath MathBroker GENSS Stilo amc CDs MK
% LocalWords:  ORCCA Eindhoven LeActiveMath Aldor CoCoA TPS PVS Phrasebooks CMU
% LocalWords:  NIST DocBook IST eLearning WebAlt IMACS ACA CML co STMML RoboML
% LocalWords:  Wiris MKM MSDL OM phrasebooks OCDs eg CAS OCD LeAM Botana FET TU
% LocalWords:  SME INRIA Antipolis INLINE Zuse Zentrum Informationstechnik ZIB
% LocalWords:  Explo Centre DFKI Programme DLMF MiKo macs mo workpackage fmpage
% LocalWords:  wpbox
