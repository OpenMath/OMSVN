\chapter{Project Overview}

Thematic Networks are designed to foster communities and coordinate
groups working in the same broad area.  The main achievements of the
OpenMath Thematic Network were as follows:

\begin{itemize}

\item It produced a new OpenMath standard which takes account of recent
advances in XML technology.

\item It increased significantly the number of available Content
Dictionaries, providing an enlarged vocabulary for machine-processable
mathematical discourse.

\item It supported European involvement in the W3C Math Working Group
and contributed substantially to the MathML~2.0 and MathML 2.0 (second
edition) Recommendations.

\item It organised a series of seven workshops devoted to mathematical
markup.

\item It provided training and developer resources via the website at
\url{http://www.openmath.org}.

\end{itemize}

The members of the Consortium were as follows:
\begin{description}

\item[NAG Ltd] NAG is Europe's largest vendor of mathematical software.
It is involved heavily in the W3C Math Working Group, providing one of
the current co-chairs.  NAG uses semantic markup to tailor documentation
for specific user communities, and to generate software components from
abstract specifications.  In addition to coordinating the Thematic
Network is ran the website and developed a range of validation tools for
OpenMath documents and Content Dictionaries.  It also led the
development of the OpenMath~2 Standard.

\item[University of Bath] The Computing Department at the University of
Bath has a long history of involvement in Computer Algebra, and until
recently was part of the School of Mathematical Sciences.  It
coordinated the development of Content Dictionaries during the project.

\item[Stilo Technology Ltd] Stilo is an SME providing XML-based services
in a broad range of areas.   It is involved in a number of W3C
activities (including MathML) and was heavily involved in the
development of OpenMath~2.

\item[INRIA Sophia-Antipolis] INRIA is one of the three original W3C
host institutions and the group at Sophia-Antipolis has been involved in
the Math Working group from its inception.  They also have a long
history in the development of mathematical software.

\item[University of St Andrews] St Andrews coordinates development of
the GAP algebraic software system, which can read and write OpenMath.
  
\item[Technical University of Eindhoven] The group at Eindhoven has been
involved in the development of electronic mathematical textbooks,
notably \emph{Algebra Interactive} for some years, and are using
OpenMath to communicate between user interfaces and a range of
mathematical engines.  They contributed a significant number of Content
Dictionaries during the project.

\item[Springer Verlag] Springer is interested in electronic and
interactive publishing, and have collaborated with Eindhoven in the
development and publication of \emph{Algebra Interactive}.
 
\item[University of Nice]  The M@INLINE Project at the University of
Nice is interested in e-learning and the development of tools for
delivering interactive courses in Mathematics.  They have developed
various OpenMath tools including an editor and a rendering agent.

\item[Konrad-Zuse Zentrum fur Informationstechnik] ZIB is one of the
main developers of the Reduce Computer Algebra System.

\item[Explo-IT Research/Nibbles.it] Exploit (later replaced in the
consortium by Nibbles.it) develops middleware for
the integration of mathematical components into other software systems.

\item[Research Institute for Symbolic Computation] RISC is one of the
largest research centres for Computer Algebra in the world.  It played a
substantial part in the development of the OpenMath~2 Standard.

\item[German Research Centre for Artificial Intelligence]  Based at the
DFKI, the ActiveMath project is developing technology for e-learning.  They
make substantial use of OMDoc, a mechanism for representing mathematical
documents which is built on OpenMath.

\item[University of Helsinki]  The University of Helsinki is also
interested in e-learning and, through its collaboration with Florida
State University, has developed a substantial body of material for
teaching undergraduate calculus.

\item[University of Cologne]  Cologne is interested in the theory behind
semantic markup for mathematics, and provided in-depth and thorough
critiques of drafts of both MathML~2 (second edition) and OpenMath~2.

\item[International University of Bremen] Bremen is the lead developer
of the OMDoc system for representing complete mathematical documents in
a machine-processable form.  They are involved in the W3C Math Working
group and were involved in the development of OpenMath~2.

\end{description}
