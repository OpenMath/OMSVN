\chapter{Outlook}

Each partner in the Consortium has benefited from involvement in the project,
and intends to exploit the results as follows.


\section{The Numerical Algorithms Group Ltd}

NAG Ltd.~develops and sells algorithmic software to an increasingly
broad range of software developers and end-users.  While the quality of
the code is of course paramount, in a commercial package the quality of
the accompanying documentation is also critical.  Nowadays the ability
to subset and re-use documentation in different contexts is also
important.  NAG has moved its documentation base from SGML+LaTeX to
XHTML with embedded OpenMath and MathML for mathematics.  The
documentation of its next product releases (expected early 2005) will,
in addition to the traditional PDF, be provided in XHTML+MathML.

NAG has extended its documentation base to provide machine-processable
specifications of its routine interfaces, parameter descriptions etc.
These specifications can then be used to generate automatically software
components to allow NAG products to be used in a range of different
environments.  One product based on this technology - Mark 7 of NAG's C
Library - is already available and is doing well in the marketplace.
A variety of further products are expected in the next twelve months.
The use of semantic markup in general --- and for mathematics in
particular --- enables NAG to produce new products and bespoke
components with far less effort and a higher degree of correctness than
would otherwise be the case.

Looking to the future NAG intends to continue its involvement in the
OpenMath Society (for which it provides the Vice-President) and the W3C
Math Interest Group (for which it provides one of the co-chairs).  NAG
is investigating the use of OpenMath and semantic web technologies,
combined with web and grid services, to provide ``algorithms on demand''
for users alongside its more traditional business model.

\section{University of Bath}

\section{Stilo Technology Ltd}

The results and achievements of the OpenMath Thematic Network project
have benefited Stilo in two related ways. In general, the experience
gained in the various tools and technologies improves the quality of
product design and implementation and also supports Stilo's professional
services and consultancy activities. More specifically, the inclusion of
project results and their architectural principles improves the product
offering.  Another very useful outcome has been the opportunity to
exchange ideas with other working groups involved in the embedding of
mathematical objects within a wider context (e.g.~OMDoc).
 
Stilo is currently implementing mathematical functionality into SophX,
its Knowledge Engineering platform. This involves the integration of
mathematical information with information from other formalisms,
including the OWL and RDF Semantic Web languages. The experience gained
on the OpenMath projects in different forms of mathematical
representations, and the preservation of semantics across
transformations, will be of great value here.


\section{INRIA Sophia-Antipolis}

INRIA benefited and hope to benefit from the OpenMath Thematic Network
in three main areas:  keeping contacts and establishing new ones in our
community, development and dissemination of our own software and
research contracts with industrial partners.
                                                                                
As can be expected for such a network, the funding permitted us to keep
close contacts with both the network partners and (for us involved in
the W3C Math working group), the MathML community. In the course of the
network life, this gave us several new academic and industrial contacts
and invitations to workshops (especially in the area of searching
mathematical data).

The developement of OpenMath 2 was important for us because we see
OpenMath as the standard that will enable our mathematical software to
be as widely used as possible.  The research group working on computer
algebra and related topics at INRIA often deals with quite specialized
subjects requiring specific implementations (for efficiency or
prototyping reasons they cannot be implemented in existing systems). For
example the CAFE research group implements some of its algorithm for
solving differential equations in the Aldor programming language.
Distribution of these programs is difficult and allowing distant access
through a Web service is a very efficient solution both for the
implementer and the potential users (that normally need to use the
programs on rare occasions). It is thus necessary to be able to exchange
the mathematical objects in some ways as standard as possible. The new
version of OpenMath produced by the Thematic Network, with its better
alignment with XML, allows the use of common tools and libraries to build
applications understanding and generating XML, which is very important
for the adoption of OpenMath. We made a fist promising experiment with
the people of the MONET European project to build a symbolic
differential equation solver as a Web service.  In the future, we intend
to provide OpenMath interfaces to our programs and spread our experience
within INRIA. In the long term this will certainly greatly improve acess
to some of our works.
                                                                                
Last but not least, the work we did thanks to the funding of the
Thematic Network made us visible for new research partners in the
industry. The French Airbus company contacted us two years ago as the
French experts on MathML and OpenMath. Looking at their underlying
problem (the development of an environment for managing aerodynamics
models), we have been able to suggest them new solutions (based on the
use of a computer algebra system instead of developing an entirely
custom application) and since then we have been working with them on two
research contracts, and we expect a new one to begin in 2005.


\section{University of St Andrews}
  
\section{Technical University of Eindhoven}

\section{Springer Verlag}
 
\section{University of Nice}

\section{Konrad-Zuse Zentrum fur Informationstechnik}

The OpenMath (and MathML) activities at the Zuse-Institute Berlin are
traditionally based in the field of computer algebra systems and their
communication. Starting in the mid 1990s, ZIB has produced modules which
allow the exchange of mathematical objects between software systems,
originally using a proprietary protocol and, in more recent years, using
standardised protocols. These activities, which allow the use of
specialised software within more general computations and parallel
computation, will play an important role in the future.

Another relevant activity is the involvement of the Zuse-Institute as a
driving force in the Math-Net project under the aegis of the
International Mathematical Union (IMU). In this framework, the
collection of mathematical information worldwide and the ability to
supply mathematicians with high-quality information is of high
importance.  Mathematical information, especially that available on the
World Wide Web, will increasingly contain information encoded in
OpenMath or MathML.  Therefore it has been crucial for the groups at the
Zuse-Institute to keep in touch with the developments in OpenMath and
W3C/Math.


\section{Explo-IT Research/Nibbles.it}

The OpenMathTN Project was for Explo-IT Research first, and for
Nibbles.it s.r.l.~later, a very interesting occasion for entering and
staying in touch with a flexible and motivated Consortium.
                                                                                
Thanks to this project the OpenMath standard has reached a further level
of maturity and dissemination.
                                                                                
Nibbles.it s.r.l.~is interested in further projects in which it can
offer its consolidated experience in Internet technologies and in
graphical development, in order to produce distributed components,
stand-alone platforms or client applications which interacts with the
mathematical world.


\section{Research Institute for Symbolic Computation}

Bruno Buchberger, in the frame of the Theorema Project, plans a ``Special
Semester on Groebner Bases'' in 2006 at the new RICAM institute (Radon
Institute for Computational and Applied Mathematics) of the Austrian
Academy of Science jointly organized with RISC (Research Institute for
Symbolic Computation) in Linz, Austria. One of the essential activities
in the frame of this special semester will be the build-up of a formal
mathematical knowledge base on Groebner bases theory and applications,
which will be publically accessible over the web. The results of the
OpenMath project will be an essential starting point as a uniform format
for the  information in this knowledge base. The knowledge base will
collect all the definitions, theorems, and algorithms of Groebner bases
theory  from the  textbooks and more than 500 research publications on
Groebner bases. This will allow translation into the various formats
used in the mathematical software systems (computer algebra and
computational logic systems) and automated processing of the information
(verification, hierarchical grouping, comparison) in these systems (in
particular, the Theorema system). The formal knowledge base will be
intimately connected with a date base of downloadable publications on
Groebner bases.

RISC is also founding member of a recently started ``Austrian Grid''
initiative, where a large scale (probably Globus-based) grid will be
established. Based on the experience on Web Service interfaces for
mathematical software acquired during the lifetime of the thematic
network, we will continue to choose OpenMath as the representation
language since it has, in our opinion, more potential for
semantic-preserving transmission of mathematical data than other XML
languages such as MathML.

\section{German Research Centre for Artificial Intelligence}

\section{University of Helsinki}

The activities of the Helsinki group focus on developing new and
innovative ways to use information technology in education.    The goal
is to automate, as much as possible, the delivery of education. OpenMath
will be the key enabling technology for the next generation of eLearning
systems.  The Helsinki group will exploit the results of this and past
OpenMath projects to develop high quality educational on-line content in
mathematics.  The first version of a virtual course in calculus will be
offered starting fall 2004.  The content developed for this on-line
course uses MathML, a limited version of OpenMath.  As tools to edit and
browse OpenMath content develop, the virtual courses developed in
Helsinki will use OpenMath.  Depending on the availability of funds the
next generation learning system, tentatively called WebALT,  and
materials benefiting from its advanced features can be taken into use as
quickly  as in 2006.  An eContent proposal, entitled WebALT,   to fund
this work is pending.  These efforts rely on OpenMath in an essential
way. 

\section{University of Cologne}

The Cologne group is working to make the theoretical and practical
experience gained with the semantic content markup of OpenMath and
MathML available to researchers outside its core application area.  As
such, the group is part of the pending WebALT eContent proposal
spear-headed by the Helsinki group, with a focus on integrating the
mathematical and linguistic aspects of the proposed project.  In another
direction, the group works towards the goal of a deeper integration of
semantic content markup with the Semantic Web's metadata markup
framework, and in particular, the goal of integrating semantic content
markup into a Semantic Grid.


\section{International University of Bremen}

